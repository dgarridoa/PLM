\documentclass[10pt]{article}
\usepackage[utf8]{inputenc}
\usepackage[activeacute,spanish,es-nodecimaldot]{babel}
\usepackage[left=1.5cm,top=1.5cm,right=1.5cm, bottom=1.5cm,letterpaper, includeheadfoot]{geometry}
\usepackage[parfill]{parskip}

\usepackage{amssymb, amsmath, amsthm}
\usepackage{graphicx}
\usepackage{lmodern,url}
\usepackage{paralist} %util para listas compactas

\usepackage{fancyhdr}
\pagestyle{fancy}
\fancypagestyle{plain}{%
\fancyhf{}
\lhead{\footnotesize\itshape\bfseries\rightmark}
\rhead{\footnotesize\itshape\bfseries\leftmark}
}

% macros
\newcommand{\QQ}{\mathbb Q}
\newcommand{\RR}{\mathbb R}
\newcommand{\NN}{\mathbb N}
\newcommand{\ZZ}{\mathbb Z}
\newcommand{\CC}{\mathbb C}

%Teoremas, Lemas, etc.
\theoremstyle{plain}
\newtheorem{teo}{Teorema}
\newtheorem{lem}{Lema}
\newtheorem{prop}{Proposición}
\newtheorem{cor}{Corolario}

\theoremstyle{definition}
\newtheorem{defi}{Definición}
\newtheorem{eje}{Ejemplo}
\newtheorem{ejer}{Ejercicio}
% fin macros

%%%%% NOMBRE AUXILIARES Y FECHA
\newcommand{\sca}{Diego Garrido}
\newcommand{\fecha}{23 de  abril  de 2020}

%%%%%%%%%%%%%%%%%%

%Macros para este documento
\newcommand{\cin}{\operatorname{cint}}

\begin{document}
%Encabezado
\fancyhead[L]{Facultad de Ciencias Físicas y Matemáticas}
\fancyhead[R]{Universidad de Chile}
\vspace*{-1.2 cm}
\begin{minipage}{0.6\textwidth}
\begin{flushleft}
\hspace*{-0.5cm}\textbf{MA4702. Programación Lineal Mixta. 2020.}\\
\hspace*{-0.5cm}\textbf{Profesor:} José Soto\\
\hspace*{-0.5cm}\textbf{Auxiliar:} \sca\\
\hspace*{-0.5cm}\textbf{Fecha:} \fecha.
\end{flushleft}
\end{minipage}
\begin{minipage}{0.36\textwidth}
\begin{flushright}
\includegraphics[scale=0.15]{fcfm}
\end{flushright}
\end{minipage}
\bigskip
%Fin encabezado

%Titulo Auxiliar
\begin{center}
\LARGE\textbf{Modelos y Dualidad}
\end{center}

\begin{table}[h]
\begin{center}
\begin{tabular}{|l|l|l|l|}
\hline
                               & min          & max          &                                \\ \hline
\multirow{}{}{Restricciones} & $\geq b_{i}$ & $\geq 0$     & \multirow{}{}{Variables}     \\ \cline{2-3}
                               & $\leq b_{i}$ & $\leq 0$     &                                \\ \cline{2-3}
                               & $= b_{i}$    & Libre        &                                \\ \hline
\multirow{}{}{Variables}     & $\geq 0$     & $\leq c_{j}$ & \multirow{}{}{Restricciones} \\ \cline{2-3}
                               & $\leq 0$     & $\geq c_{j}$ &                                \\ \cline{2-3}
                               & Libre        & $= c_{j}$    &                                \\ \hline
\end{tabular}
\end{center}
\end{table}

\section{Minium Spanning Tree (MST)}
Queremos construir una red de comunicación que conecte a todas las ciudades a costo mínimo, para ello contamos con un grafo conectado no dirigido $G(V,E)$, donde $V$ es el conjunto de ciudades, $E$ las carreteras que conectan las ciudades y $w_{e}$ costo de usar la carretera $e \in E$. El problema anterior se puede formular como un MST.

\begin{itemize}
\item[a)] Considere los siguientes modelos para el MST:

\begin{table}[h]
\begin{tabular}{ll}
\begin{equation*}
\boxed{
\begin{aligned}
\begin{center}\text{modeloConect}\end{center}\\
& \min\        & \sum_{e \in E} w_{e}x_{e} \\
& \text{s.a. } &  x(E) =& |V|-1 \\
& & x(\delta(S))  \geq  &1   \quad \forall S \subset V, \quad S\neq \emptyset\\
 &&  x_{e}  \in &\{0,1\} \quad  \forall e \in E\\
\end{aligned}
}
\end{equation*}
 & \begin{equation*}
\boxed{
\begin{aligned}
\begin{center}\text{modeloCiclos}\end{center}\\
& \min\        & \sum_{e \in E} w_{e}x_{e} \\
& \text{s.a. }  &x(E) =& |V|-1 \\
 & & x(E(S))  \leq&  |S|-1 \quad \forall S \subset V, \quad S\neq \emptyset\\
 &&  x_{e} \in & \{0,1\} \quad  \forall e \in E\\
\end{aligned}
}
\end{equation*}
\end{tabular}
\end{table}

Demuestre que ambos son exactos

\item[b)] Relaje la integralidad de la primera formulación. Demuestre que no es un modelo exacto, pero si un modelo.

\end{itemize}

\section{Maximum Set Packing (MSP)}

 Dado un universo $U$ y una familia $S$ de subconjuntos de $U$, un empaquetamiento es una subfamilia $C\subset S$ tal que todos los conjuntos en $C$ son disjuntos de a pares (en otras palabras, no hay dos conjuntos que compartan un elemento), siendo el tamaño del empaquetamiento igual a $|C|$. EL problema de empaquetamiento busca maximizar el número de conjuntos disjuntos de a pares en $S$ que se puede escoger.

%https://en.wikipedia.org/wiki/Set_packing
%terminar de formular y entender
%https://en.wikipedia.org/wiki/Set_cover_problem
\begin{center}
\begin{alignat*}{5}
&\max\ & \sum_{s \in S}x_{s}\\
&\text{s.a. } &  \sum_{s \in S:e \in S}x_{s}  \leq  &1 &\quad \forall e \in U\\
&& x_{s}  \in  &\{0,1\} &\quad \forall s \in S
\end{alignat*}
\end{center}

Obtenga el dual de la relajación lineal, luego restringa el dominio de las variables a $\{0,1\}$ e interprete el problema obtenido.

\section{Single Source Capacitated Facility Location Problem (CFLP)}
El problema consiste en escoger donde instalar bodegas tal que se satisface la demanda de los minoristas a costo mínimo. Cada minorista debe estar asignado a una única bodega y esta debe satisfacer toda su demanda por producto único sin violar su restricción de capacidad.

Parámetros del problema:
\begin{itemize}
    \item $M$: sitios potenciales, con $M = \{1, \ldots, m\}$
    \item $N$: minoristas, con $N = \{1, \ldots, n\}$
    \item $q_{j}$: capacidad de bodega $j$.
    \item $w_{i}$: demanda del minorista $i$.
    \item $f_{j}$: costo de abrir bodega en sitio $j$.
    \item $c_{ij}$: costo de transportar $w_{i}$ unidades de producto de bodega $j$ a minorista $i$.
\end{itemize}

\begin{itemize}
    \item[a)] Formule el PLB.
    \item[b)] Obtenga el dual de la relajación lineal.
\end{itemize}

\end{document}
