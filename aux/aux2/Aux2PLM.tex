\documentclass[10pt]{article}
\usepackage[utf8]{inputenc}
\usepackage[activeacute,spanish,es-nodecimaldot]{babel}
\usepackage[left=1.5cm,top=1.5cm,right=1.5cm, bottom=1.5cm,letterpaper, includeheadfoot]{geometry}
\usepackage[parfill]{parskip}

\usepackage{amssymb, amsmath, amsthm}
\usepackage{graphicx}
\usepackage{lmodern,url}
\usepackage{paralist} %util para listas compactas

\usepackage{fancyhdr}
\pagestyle{fancy}
\fancypagestyle{plain}{%
\fancyhf{}
\lhead{\footnotesize\itshape\bfseries\rightmark}
\rhead{\footnotesize\itshape\bfseries\leftmark}
}


% macros
\newcommand{\QQ}{\mathbb Q}
\newcommand{\RR}{\mathbb R}
\newcommand{\NN}{\mathbb N}
\newcommand{\ZZ}{\mathbb Z}
\newcommand{\CC}{\mathbb C}

%Teoremas, Lemas, etc.
\theoremstyle{plain}
\newtheorem{teo}{Teorema}
\newtheorem{lem}{Lema}
\newtheorem{prop}{Proposición}
\newtheorem{cor}{Corolario}

\theoremstyle{definition}
\newtheorem{defi}{Definición}
\newtheorem{eje}{Ejemplo}
\newtheorem{ejer}{Ejercicio}
% fin macros

%%%%% NOMBRE AUXILIARES Y FECHA
\newcommand{\sca}{Diego Garrido}
\newcommand{\fecha}{30 de  abril  de 2020}

%%%%%%%%%%%%%%%%%%

%Macros para este documento
\newcommand{\cin}{\operatorname{cint}}


\begin{document}
%Encabezado
\fancyhead[L]{Facultad de Ciencias Físicas y Matemáticas}
\fancyhead[R]{Universidad de Chile}
\vspace*{-1.2 cm}
\begin{minipage}{0.6\textwidth}
\begin{flushleft}
\hspace*{-0.5cm}\textbf{MA4702. Programación Lineal Mixta. 2020.}\\
\hspace*{-0.5cm}\textbf{Profesor:} José Soto\\
\hspace*{-0.5cm}\textbf{Auxiliar:} \sca\\
\hspace*{-0.5cm}\textbf{Fecha:} \fecha.
\end{flushleft}
\end{minipage}
\begin{minipage}{0.36\textwidth}
\begin{flushright}
\includegraphics[scale=0.15]{fcfm}
\end{flushright}
\end{minipage}
\bigskip
%Fin encabezado

%Titulo Auxiliar
\begin{center}
\LARGE\textbf{Dualidad}
\end{center}

\begin{table}[h]
\begin{center}
\begin{tabular}{|l|l|l|l|}
\hline
                               & min          & max          &                                \\ \hline
\multirow{}{}{Restricciones} & $\geq b_{i}$ & $\geq 0$     & \multirow{}{}{Variables}     \\ \cline{2-3}
                               & $\leq b_{i}$ & $\leq 0$     &                                \\ \cline{2-3}
                               & $= b_{i}$    & Libre        &                                \\ \hline
\multirow{}{}{Variables}     & $\geq 0$     & $\leq c_{j}$ & \multirow{}{}{Restricciones} \\ \cline{2-3}
                               & $\leq 0$     & $\geq c_{j}$ &                                \\ \cline{2-3}
                               & Libre        & $= c_{j}$    &                                \\ \hline
\end{tabular}
\begin{tabular}{|l|l|l|l|}
\hline
\textbf{Primal/Dual}   & \textbf{Optimo finito} & \textbf{No acotado} & \textbf{Infactible} \\ \hline
\textbf{Optimo finito} & Posible                & Imposible           & Imposible           \\ \hline
\textbf{No acotado}    & Imposible              & Imposible           & Posible             \\ \hline
\textbf{Infactible}    & Imposible              & Posible             & Posible             \\ \hline
\end{tabular}
\end{center}
\end{table}

\section{Lema de Farkas}
Pruebe otras versiones del lema de Farkas:

\begin{itemize}
\item[a)] $\{Ax=b, \; x\geq0\}\neq\emptyset \iff \{A^{T}y\leq0, \; b^{T}y>0\}=\emptyset$

\item[b)] $\{Ax\leq0, \; x\geq0, \; c^{T}x>0\}\neq\emptyset \iff \{A^{T}y\geq c, \; y\geq0\}=\emptyset$
\end{itemize}

\section{Dualidad y relajación Lagrangeana}
Consideré el siguiente problema primal:

\begin{center}
\begin{alignat*}{5}
\min\  c^{T}x\\
\text{s.a. }  Ax&\leq b\\
x&\geq 0
\end{alignat*}
\end{center}
Demuestre que la mejor cota (cota inferior más cercana al valor óptimo del primal) lagrangeana del primal es su dual. \textit{Hint}: Escriba la relajación lagrangeana del primal e imponga condiciones sobre los multiplicadores para que sea una cota inferior distinta de $-\infty$.

\section{Maximum Flow Problem}

Considere el grafo dirigido $G(V,E)$, el objetivo del problema de flujo máximo es enviar la mayor cantidad de flujo desde un nodo $s$ a un nodo $t$, donde los arcos tienen capacidades positivas $c = (c_{e})_{e \in E}$.
\begin{itemize}
    \item[a)] Formule el PL y obtenga su dual
    \item[b)] Obtenga el dual usando relajación lagrangeana
\end{itemize}

\section{Teorema Carathéodory}
Sea $P \subset \mathb{R}^{n}$ un politopo y $W=\{x^{1},\ldots, x^{k}\}$ sus puntos extremos.

\begin{itemize}
    \item[a)] Demuestre que $P=conv(W)$.
    \item[b)] Muestre que todo elemento de $P$ puede ser expresado como una combinación convexa de a lo más $n+1$ puntos extremos. \textit{Hint:} planteé el poliedro asociado a un punto cualquiera de $P$.
\end{itemize}

\end{document}
