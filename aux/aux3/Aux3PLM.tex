\documentclass[10pt]{article}
\usepackage[utf8]{inputenc}
\usepackage[activeacute,spanish,es-nodecimaldot]{babel}
\usepackage[left=1.5cm,top=1.5cm,right=1.5cm, bottom=1.5cm,letterpaper, includeheadfoot]{geometry}
\usepackage[parfill]{parskip}

\usepackage{amssymb, amsmath, amsthm}
\usepackage{graphicx}
\usepackage{lmodern,url}
\usepackage{paralist} %util para listas compactas

\usepackage{fancyhdr}
\pagestyle{fancy}
\fancypagestyle{plain}{%
\fancyhf{}
\lhead{\footnotesize\itshape\bfseries\rightmark}
\rhead{\footnotesize\itshape\bfseries\leftmark}
}


% macros
\newcommand{\QQ}{\mathbb Q}
\newcommand{\RR}{\mathbb R}
\newcommand{\NN}{\mathbb N}
\newcommand{\ZZ}{\mathbb Z}
\newcommand{\CC}{\mathbb C}

%Teoremas, Lemas, etc.
\theoremstyle{plain}
\newtheorem{teo}{Teorema}
\newtheorem{lem}{Lema}
\newtheorem{prop}{Proposición}
\newtheorem{cor}{Corolario}

\theoremstyle{definition}
\newtheorem{defi}{Definición}
\newtheorem{eje}{Ejemplo}
\newtheorem{ejer}{Ejercicio}
% fin macros

%%%%% NOMBRE AUXILIARES Y FECHA
\newcommand{\sca}{Diego Garrido}
\newcommand{\fecha}{14 de mayo  de 2020}

%%%%%%%%%%%%%%%%%%

%Macros para este documento
\newcommand{\cin}{\operatorname{cint}}


\begin{document}
%Encabezado
\fancyhead[L]{Facultad de Ciencias Físicas y Matemáticas}
\fancyhead[R]{Universidad de Chile}
\vspace*{-1.2 cm}
\begin{minipage}{0.6\textwidth}
\begin{flushleft}
\hspace*{-0.5cm}\textbf{MA4702. Programación Lineal Mixta. 2020.}\\
\hspace*{-0.5cm}\textbf{Profesor:} José Soto\\
\hspace*{-0.5cm}\textbf{Auxiliar:} \sca\\
\hspace*{-0.5cm}\textbf{Fecha:} \fecha.
\end{flushleft}
\end{minipage}
\begin{minipage}{0.36\textwidth}
\begin{flushright}
\includegraphics[scale=0.15]{fcfm}
\end{flushright}
\end{minipage}
\bigskip
%Fin encabezado

%Titulo Auxiliar
\begin{center}
\LARGE\textbf{Dimensión y Caras}
\end{center}


\section{P1}
Un vertex cover ($VC$) de $G=(V,E)$ es un conjunto de vértices $W\subseteq V$ talque para cada $e \in E$ tiene al menos un extremo en $W$. Sea $P_{vc}(G)=conv\{\chi^{W}:W \text{es } VC \text{ de } G\} \subseteq\mathbb{R}^{V}$. Es fácil ver que:

\begin{align*}
  P_{vc}(G)\subseteq Q(G):=\{x\in\mathbb{R}^{V}:x_{u}+x_{v}\geq 1, \forall (u,v) \in E, 0\leq x_{v} \leq 1, \forall v \in V\}
\end{align*}

\begin{itemize}
\item[a)] Pruebe que $P_{vc}$ es de dimensión completa.

\item[b)] Pruebe que las desigualdades $x_{v}\leq1$ inducen facetas.

\item[c)] Pruebe que si existe un ciclo $\{(u,v), (v,w), (w,u)\}\subseteq E$ de largo 3, entonces, la desigualdad $x_{u}+x_{v}+x_{w}\geq2$ es válida y la desigualdad $x_{u}+x_{v}\geq1$ no induce faceta.

\end{itemize}

\section{P2}
Sea $a\in \mathbb{R}^{n}_{++}$ y $b\in\mathbb{R}_{++}$ tal que $\sum_{i=1}^{n}a_{i}>b$. Consideremos el polítopo de Knapsack:

\begin{align*}
K_{n} = conv(\{x\in\{0,1\}^{n}:a^{T}x\leq b\})
\end{align*}

Y el polítpo de Knapsack Fraccionario:

\begin{align*}
K\text{-}Fr_{n} = \{x\in[0,1]^{n}:a^{T}x\leq b\}
\end{align*}

\begin{itemize}
\item[a)] Contraste $K_{n}$ con $K\text{-}Fr_{n}$. ¿$K_{n}=K\text{-}Fr_{n}$? ¿Se tiene alguna inclusión?
\item[b)] Calcula la dimensión de $K\text{-}Fr_{n}$ y de $K_{n}$.
\item[c)] Encuentre las facetas de $K\text{-}Fr_{n}$.
\end{itemize}


\end{document}
