\documentclass[10pt]{article}
\usepackage[utf8]{inputenc}
\usepackage[activeacute,spanish,es-nodecimaldot]{babel}
\usepackage[left=1.5cm,top=1.5cm,right=1.5cm, bottom=1.5cm,letterpaper, includeheadfoot]{geometry}
\usepackage[parfill]{parskip}

\usepackage{amssymb, amsmath, amsthm}
\usepackage{graphicx}
\usepackage{lmodern,url}
\usepackage{paralist} %util para listas compactas

\usepackage{fancyhdr}
\pagestyle{fancy}
\fancypagestyle{plain}{%
\fancyhf{}
\lhead{\footnotesize\itshape\bfseries\rightmark}
\rhead{\footnotesize\itshape\bfseries\leftmark}
}


% macros
\newcommand{\QQ}{\mathbb Q}
\newcommand{\RR}{\mathbb R}
\newcommand{\NN}{\mathbb N}
\newcommand{\ZZ}{\mathbb Z}
\newcommand{\CC}{\mathbb C}

%Teoremas, Lemas, etc.
\theoremstyle{plain}
\newtheorem{teo}{Teorema}
\newtheorem{lem}{Lema}
\newtheorem{prop}{Proposición}
\newtheorem{cor}{Corolario}

\theoremstyle{definition}
\newtheorem{defi}{Definición}
\newtheorem{eje}{Ejemplo}
\newtheorem{ejer}{Ejercicio}
% fin macros

%%%%% NOMBRE AUXILIARES Y FECHA
\newcommand{\sca}{Diego Garrido}
\newcommand{\fecha}{11 de junio  de 2020}

%%%%%%%%%%%%%%%%%%

%Macros para este documento
\newcommand{\cin}{\operatorname{cint}}


\begin{document}
%Encabezado
\fancyhead[L]{Facultad de Ciencias Físicas y Matemáticas}
\fancyhead[R]{Universidad de Chile}
\vspace*{-1.2 cm}
\begin{minipage}{0.6\textwidth}
\begin{flushleft}
\hspace*{-0.5cm}\textbf{MA4702. Programación Lineal Mixta. 2020.}\\
\hspace*{-0.5cm}\textbf{Profesor:} José Soto\\
\hspace*{-0.5cm}\textbf{Auxiliar:} \sca\\
\hspace*{-0.5cm}\textbf{Fecha:} \fecha.
\end{flushleft}
\end{minipage}
\begin{minipage}{0.36\textwidth}
\begin{flushright}
\includegraphics[scale=0.15]{fcfm}
\end{flushright}
\end{minipage}
\bigskip
%Fin encabezado

%Titulo Auxiliar
\begin{center}
\LARGE\textbf{Generación de columnas}
\end{center}


\section{Vehicle Routing Problem (VRP)}

Cuenta con una flota homogenea de $n$ vehículos de capacidad $F$ con la que debe satisfacer la demanda de $m$ clientes a costo mínimo, siendo la demanda $d_{i}$ para $i\in\{1,\ldots, m\}$. Sea $G(V,E)$ el grafo dirigido del problema, con $V=\{0\}\cup[m]$ y $E=V\times V$, el nodo $\{0\}$ representa el centro de distribución (CD) de donde todos los vehículos empiezan y terminan su ruta, por último, sea $c_{e}$ el costo de usar el arco $e\in E$.  

\begin{itemize}
\item[a)] Formule el problema anterior como un PE que solo use la variable binaria $x_{ij}^{k}$ que toma valor $1$ y el vehículo $k$ usa el arco $(i,j)$, 0 sino.\\

\textbf{Solución:}

\begin{align}
\min &\sum_{k=1}^{n}\sum_{i=0}^{m}\sum_{j=0}^{m} x_{ij}^{k}\nonumber\\ 
& \sum_{k=1}^{n}\sum_{i=0}^{m} x_{ij}^{k} = 1, \;\forall j \in [m]\\
& \sum_{j=1}^{m}x_{0j}^{k}=\sum_{i=1}^{m}x_{i0}^{k}=1, \; \forall k \in [n]\\
& \sum_{(i,j)\in \delta(S)}x_{ij}^{k}\leq |S|-1, \forall k\in[n], S\subseteq [m], S\neq\emptyset\\
& \sum_{i=1}^{m}\sum_{j=0}^{m} d_{i}x_{ij}^{k}\leq F \; \forall k \in [n]\\
& x_{ij}^{k}\in\{0,1\} \; \forall (i,j)\in E, k \in [n]\nonumber
\end{align}

La restricción (1) indica que cada cliente es atendido por un único vehículo y solo una vez, (2) obliga a que todo vehículo debe salir del CD hacia un cliente y volver desde un cliente al CD, (3) corresponde a la restricción de eliminación de subtour, así las rutas de los vehículos son conexas, por último, (4) corresponde a la restricción de capacidad de los vehículos. Es importante notar que por (1) se tiene que $\sum_{i=0}^{m}x_{ij}^{k}\leq 1, \; \forall k \in [n], j \in [m]$.

\item[b)] Formule el problema como un PE que solo use la variable $x_{p}$ que toma valor 1 si se usa la ruta $p\in P$.\\ \textbf{Indicación}: una ruta $p\in P$ pertenece a $\{0,1\}^{E}$, además, defina explícitamente $c_{p}$ como el costo de usar la ruta $p$.

\textbf{Solución:}

El costo de un patrón $p$ es $c_{p}=\sum_{i=0}^{m}\sum_{j=0}^{m}c_{ij}p_{ij}$, notar que como el patrón representa una ruta cumple $p_{j}=\sum_{i=0}^{m}p_{ij}\leq 1, \;\forall j \in [m]$.

\begin{align}
\min & \sum_{p\in P}c_{p}x_{p}\nonumber\\
& \sum_{p\in P}x_{p}=n\\
& \sum_{p\in P}p_{j}x_{p} = 1 \; \forall j \in [m]\\
& x_{p}\in\{0,1\} \; \forall p \in P\nonumber
\end{align}
La restricción (5) obliga a que la cantidad de rutas a utilizar tiene que ser exactamente el tamaño de la flota y por (6) se obliga a que cada cliente es atendido por una única ruta.

\item[c)] Para $Q\subseteq P$, formule el master problem MP(Q) y su DUAL-MP(Q).

\textbf{Solución:}

Para $Q\subset P$ el problema maestro se escribe como sigue:
\begin{align}
    (MP(Q))\min & \sum_{p\in P}c_{p}x_{p}\nonumber\\
    & \sum_{p\in Q}x_{p}=n\\
    & \sum_{p\in Q}p_{j}x_{p} = 1 \; \forall j \in [m]\\
    & x_{p}\geq 0 \; \forall p \in Q\nonumber
\end{align}

Luego su dual es:
\begin{align}
    (MP(Q))\max & \;z+\sum_{j=1}^{m}y_{j}\nonumber\\
    & z+\sum_{j=1}p_{j}y_{j}\leq c_{p} \forall p \in Q\\
    & z \; libre, y_{j} \; libre \;\forall j \in [m]\nonumber
\end{align}

, donde $z$ es la variable dual de la restricción (7) e $y$ la variable dual de la restricción (6).

\item[d)] Para una solución dual factible de DUAL-MP(Q) formule el pricing problem.

\textbf{Solución:}
El pricing problem $PP(y,z)$ corresponde a buscar la ruta $p$ que más viola que más viola (9), equivalentemente es el que viola más $z\leq c_{p}-\sum_{j=1}^{m}p_{j}y_{j}$, luego como $z$ es fijo y solo el lado derecho depende de $p$ el $PP(y,z)$ equivale a minimizar el lado derecho sujeto a las restricciones que hacen que $p$ sea una ruta (10) y (11).

\begin{align}
    \min & \sum_{i=0}^{m}\sum_{j=0}^{m}c_{ij}p_{ij}-\sum_{j=1}^{m}y_{j}\sum_{i=0}^{m}p_{ij}\nonumber\\
    & \sum_{j=1}^{m}p_{0j}=\sum_{i=1}^{m}p_{i0}=1\\
    & \sum_{(i,j)\in \delta(S)}p_{ij}\leq |S|-1, \; \forall S\subseteq [m], S\neq\emptyset\\
    & p_{ij}\in\{0,1\} \;\forall (i,j) \in E\nonumber
\end{align}
\end{itemize}

Sea $\eta$ el valor óptimo del problema anterior y $p$ la solución óptima, luego $p$ se agrega a $Q$ solo si $\eta<z$. 

\end{document}
