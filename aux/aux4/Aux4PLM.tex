\documentclass[10pt]{article}
\usepackage[utf8]{inputenc}
\usepackage[activeacute,spanish,es-nodecimaldot]{babel}
\usepackage[left=1.5cm,top=1.5cm,right=1.5cm, bottom=1.5cm,letterpaper, includeheadfoot]{geometry}
\usepackage[parfill]{parskip}

\usepackage{amssymb, amsmath, amsthm}
\usepackage{graphicx}
\usepackage{lmodern,url}
\usepackage{paralist} %util para listas compactas

\usepackage{fancyhdr}
\pagestyle{fancy}
\fancypagestyle{plain}{%
\fancyhf{}
\lhead{\footnotesize\itshape\bfseries\rightmark}
\rhead{\footnotesize\itshape\bfseries\leftmark}
}


% macros
\newcommand{\QQ}{\mathbb Q}
\newcommand{\RR}{\mathbb R}
\newcommand{\NN}{\mathbb N}
\newcommand{\ZZ}{\mathbb Z}
\newcommand{\CC}{\mathbb C}

%Teoremas, Lemas, etc.
\theoremstyle{plain}
\newtheorem{teo}{Teorema}
\newtheorem{lem}{Lema}
\newtheorem{prop}{Proposición}
\newtheorem{cor}{Corolario}

\theoremstyle{definition}
\newtheorem{defi}{Definición}
\newtheorem{eje}{Ejemplo}
\newtheorem{ejer}{Ejercicio}
% fin macros

%%%%% NOMBRE AUXILIARES Y FECHA
\newcommand{\sca}{Diego Garrido}
\newcommand{\fecha}{11 de junio  de 2020}

%%%%%%%%%%%%%%%%%%

%Macros para este documento
\newcommand{\cin}{\operatorname{cint}}


\begin{document}
%Encabezado
\fancyhead[L]{Facultad de Ciencias Físicas y Matemáticas}
\fancyhead[R]{Universidad de Chile}
\vspace*{-1.2 cm}
\begin{minipage}{0.6\textwidth}
\begin{flushleft}
\hspace*{-0.5cm}\textbf{MA4702. Programación Lineal Mixta. 2020.}\\
\hspace*{-0.5cm}\textbf{Profesor:} José Soto\\
\hspace*{-0.5cm}\textbf{Auxiliar:} \sca\\
\hspace*{-0.5cm}\textbf{Fecha:} \fecha.
\end{flushleft}
\end{minipage}
\begin{minipage}{0.36\textwidth}
\begin{flushright}
\includegraphics[scale=0.15]{fcfm}
\end{flushright}
\end{minipage}
\bigskip
%Fin encabezado

%Titulo Auxiliar
\begin{center}
\LARGE\textbf{Generación de columnas}
\end{center}


\section{Vehicle Routing Problem (VRP)}

Cuenta con una flota homogenea de $n$ vehículos de capacidad $F$ con la que debe satisfacer la demanda de $m$ clientes a costo mínimo, siendo la demanda $d_{i}$ para $i\in\{1,\ldots, m\}$. Sea $G(V,E)$ el grafo dirigido del problema, con $V=\{0\}\cup[m]$ y $E=V\times V$, el nodo $\{0\}$ representa el centro de distribución (CD) de donde todos los vehículos empiezan y terminan su ruta, por último, sea $c_{e}$ el costo de usar el arco $e\in E$.  

\begin{itemize}
\item[a)] Formule el problema anterior como un PE que solo use la variable binaria $x_{ij}^{k}$ que toma valor $1$ y el vehículo $k$ usa el arco $(i,j)$, 0 sino.

\item[b)] Formule el problema como un PE que solo use la variable $x_{p}$ que toma valor 1 si se usa la ruta $p\in P$.\\ \textbf{Indicación}: una ruta $p\in P$ pertenece a $\{0,1\}^{E}$, además, defina explícitamente $c_{p}$ como el costo de usar la ruta $p$.

\item[c)] Para $Q\subseteq P$, formule el master problem MP(Q) y su DUAL-MP(Q).
\item[d)] Para una solución dual factible de DUAL-MP(Q) formule el pricing problem.

\end{itemize}


\end{document}
