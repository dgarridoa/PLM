\documentclass[10pt]{article}
\usepackage[utf8]{inputenc}
\usepackage[activeacute,spanish,es-nodecimaldot]{babel}
\usepackage[left=1.5cm,top=1.5cm,right=1.5cm, bottom=1.5cm,letterpaper, includeheadfoot]{geometry}
\usepackage[parfill]{parskip}

\usepackage{amssymb, amsmath, amsthm}
\usepackage{graphicx}
\usepackage{lmodern,url}
\usepackage{paralist} %util para listas compactas

\usepackage{fancyhdr}
\pagestyle{fancy}
\fancypagestyle{plain}{%
\fancyhf{}
\lhead{\footnotesize\itshape\bfseries\rightmark}
\rhead{\footnotesize\itshape\bfseries\leftmark}
}


% macros
\newcommand{\QQ}{\mathbb Q}
\newcommand{\RR}{\mathbb R}
\newcommand{\NN}{\mathbb N}
\newcommand{\ZZ}{\mathbb Z}
\newcommand{\CC}{\mathbb C}

%Teoremas, Lemas, etc.
\theoremstyle{plain}
\newtheorem{teo}{Teorema}
\newtheorem{lem}{Lema}
\newtheorem{prop}{Proposición}
\newtheorem{cor}{Corolario}

\theoremstyle{definition}
\newtheorem{defi}{Definición}
\newtheorem{eje}{Ejemplo}
\newtheorem{ejer}{Ejercicio}
% fin macros

%%%%% NOMBRE AUXILIARES Y FECHA
\newcommand{\sca}{Diego Garrido}
\newcommand{\fecha}{25 de junio  de 2020}

%%%%%%%%%%%%%%%%%%

%Macros para este documento
\newcommand{\cin}{\operatorname{cint}}


\begin{document}
%Encabezado
\fancyhead[L]{Facultad de Ciencias Físicas y Matemáticas}
\fancyhead[R]{Universidad de Chile}
\vspace*{-1.2 cm}
\begin{minipage}{0.6\textwidth}
\begin{flushleft}
\hspace*{-0.5cm}\textbf{MA4702. Programación Lineal Mixta. 2020.}\\
\hspace*{-0.5cm}\textbf{Profesor:} José Soto\\
\hspace*{-0.5cm}\textbf{Auxiliar:} \sca\\
\hspace*{-0.5cm}\textbf{Fecha:} \fecha.
\end{flushleft}
\end{minipage}
\begin{minipage}{0.36\textwidth}
\begin{flushright}
\includegraphics[scale=0.15]{fcfm}
\end{flushright}
\end{minipage}
\bigskip
%Fin encabezado

%Titulo Auxiliar
\begin{center}
\LARGE\textbf{Total Unimodularidad (TU)}
\end{center}


\section{Teorema de Braum-Trotter}

Diremos que un poliedro $P$ tiene la propiedad de descomposición entera si $\forall k \in \mathbb{Z}^{+}\setminus\{0\}$ se tiene que todo vector integral del conjunto $kP$ es suma de $k$ vectores integrales de $P$. Sea $P=\{x\in\mathbb{R}^{n}: Ax\leq b, x\geq 0\}$ un poliedro racional y puntiagudo.
\begin{itemize}
    \item[a)] Demuestre que si $P$ tiene la propiedad de descomposición entera, entonces $P$ es un poliedro entero.
    \item[b)] Demuestre que si $A$ es total unimodular y $b$ es integral, entonces $P$ tiene la propiedad de descomposición entera.
    \item[c)] Concluya quue $P$ tiene la propiedad de descomposición entera para todo $b$ integral si y sólo si $A$ es total unimodular.
\end{itemize}

\section{Propiedades de una Matriz TU}
Prueba que si $A\in\{-1,0,1\}^{m\times n}$ es unimodular, entonces:
\begin{itemize}
    \item[a)] Cada submatriz cuadrada invertible de $A$ tiene una fila con un número impar de coordenadas no nulas.
    \item[b)] Si $B$ es submatriz cuadrada de $A$ tal que la suma sobre las filas y sobre las columnas de $B$ es par, entonces la suma sobre todo $B$ es divisible por 4.
\end{itemize}
\textbf{Hint: Si A es TU, entonces todo submatriz B de A es TU. Use el teorema de Ghouila-Houri.}


\end{document}
