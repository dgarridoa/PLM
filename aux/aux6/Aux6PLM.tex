\documentclass[10pt]{article}
\usepackage[utf8]{inputenc}
\usepackage[activeacute,spanish,es-nodecimaldot]{babel}
\usepackage[left=1.5cm,top=1.5cm,right=1.5cm, bottom=1.5cm,letterpaper, includeheadfoot]{geometry}
\usepackage[parfill]{parskip}

\usepackage{amssymb, amsmath, amsthm}
\usepackage{graphicx}
\usepackage{lmodern,url}
\usepackage{paralist} %util para listas compactas

\usepackage{fancyhdr}
\pagestyle{fancy}
\fancypagestyle{plain}{%
\fancyhf{}
\lhead{\footnotesize\itshape\bfseries\rightmark}
\rhead{\footnotesize\itshape\bfseries\leftmark}
}


% macros
\newcommand{\QQ}{\mathbb Q}
\newcommand{\RR}{\mathbb R}
\newcommand{\NN}{\mathbb N}
\newcommand{\ZZ}{\mathbb Z}
\newcommand{\CC}{\mathbb C}

%Teoremas, Lemas, etc.
\theoremstyle{plain}
\newtheorem{teo}{Teorema}
\newtheorem{lem}{Lema}
\newtheorem{prop}{Proposición}
\newtheorem{cor}{Corolario}

\theoremstyle{definition}
\newtheorem{defi}{Definición}
\newtheorem{eje}{Ejemplo}
\newtheorem{ejer}{Ejercicio}
% fin macros

%%%%% NOMBRE AUXILIARES Y FECHA
\newcommand{\sca}{Diego Garrido}
\newcommand{\fecha}{9 de julio  de 2020}

%%%%%%%%%%%%%%%%%%

%Macros para este documento
\newcommand{\cin}{\operatorname{cint}}


\begin{document}
%Encabezado
\fancyhead[L]{Facultad de Ciencias Físicas y Matemáticas}
\fancyhead[R]{Universidad de Chile}
\vspace*{-1.2 cm}
\begin{minipage}{0.6\textwidth}
\begin{flushleft}
\hspace*{-0.5cm}\textbf{MA4702. Programación Lineal Mixta. 2020.}\\
\hspace*{-0.5cm}\textbf{Profesor:} José Soto\\
\hspace*{-0.5cm}\textbf{Auxiliar:} \sca\\
\hspace*{-0.5cm}\textbf{Fecha:} \fecha.
\end{flushleft}
\end{minipage}
\begin{minipage}{0.36\textwidth}
\begin{flushright}
\includegraphics[scale=0.15]{fcfm}
\end{flushright}
\end{minipage}
\bigskip
%Fin encabezado

%Titulo Auxiliar
\begin{center}
\LARGE\textbf{Total Dual Integral (TDI)}
\end{center}

Sea $G=(V,E)$ un grafo conexo y no dirigido. 
El objetivo de este problema es probar que el sistema que define al polítopo de los bosques de $G$,
$B(G)=\{x\in \RR^E\colon x(E(S))\leq |S|-1, \forall\; \emptyset\neq S\subseteq V, x\geq 0\}$ es TDI.

\hspace{-15pt}\textbf{(a)} 
Escriba el dual (D) del problema $\max\{c^Tx\colon x\in B(G)\}$, usando variables $\{y_S\}_{S\subseteq V, S\neq \emptyset}$.

\hspace{-15pt}\textbf{(b)} Considere una solución dual óptima $y^*$ que minimice la cantidad $\Psi(y)=\sum_{S\subseteq V, S\neq \emptyset} y_S |S||V\setminus S|$
y pruebe que el soporte $\mathcal{L}$ de $y^*$ es una familia \textbf{laminar}, es decir que no existen dos conjuntos $A, B$ intersectantes en su soporte\footnote{$A, B$ se dicen intersectantes si $A\setminus B$, $B\setminus A$, $A\cap B$ son todos no vacíos.}. ¡Cuidado! Recuerde que no existe la variable $y_\emptyset$.\\

\hspace{-15pt}\textbf{(c)} Sea $W$ un conjunto finito, $\mathcal{L}\subseteq 2^W$ una familia laminar y $\mathcal{E}\subseteq 2^W$ una familia de conjuntos. Pruebe que la matriz $M\in \{0,1\}^{\mathcal{E}\times \mathcal{L}}$ dada por
$M_{I,J}=\begin{cases} 1 & \text{ si } I \subseteq J,\\ 0 &\text{ en otro caso,}\end{cases}$ es totalmente unimodular (TU)

\textbf{Indicación:} Use Ghouila-Houri asignando signos en las columnas adecuadamente.

\hspace{-15pt}\textbf{(d)} Usando las partes (b) y (c), pruebe que el sistema que define $B(G)$ es totalmente dual integral (TDI). Concluya que $B(G)$ es integral.

\end{document}

