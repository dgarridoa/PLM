% BEGIN_FOLD preamble

\documentclass{article}
\usepackage[activeacute,spanish]{babel}
\usepackage{lmodern}
\usepackage[T1]{fontenc}
\usepackage{url,multirow}
\usepackage[centertableaux,smalltableaux]{ytableau}
\usepackage[utf8]{inputenc}
\usepackage{cancel, comment}
\usepackage[left=2cm,top=1.5cm,right=2cm, bottom=1.5cm,letterpaper, includeheadfoot]{geometry}
\usepackage{amssymb, amsmath, amsthm, mathtools}
\usepackage{graphicx}
\usepackage{hyperref}
\hypersetup{
	colorlinks,
	linkcolor={red!50!black},
	citecolor={blue!50!black},
	urlcolor={blue!80!black}
}

\usepackage[prependcaption,textsize=tiny,,textwidth=5cm]{todonotes}
\newcommand{\js}[1]{\todo[inline,linecolor=red,backgroundcolor=red!25,bordercolor=red]{jsoto. #1}}

\usepackage{paralist}
\usepackage{contour}


%%%Definiciones
\newcommand{\dis}{\displaystyle}
\newcommand{\IV}[1]{[\![#1]\!]} %Iverson

\newcommand{\E}{\mathcal{E}}
\newcommand{\V}{\mathcal{V}}

\def\multiset#1#2{\ensuremath{
		\mathchoice{\left(\kern-.3em\left(\genfrac{}{}{0pt}{}{#1}{#2}\right)\kern-.3em\right)}}
	{\big(\!\binom{#1}{#2}\!\big)}{\big(\!\binom{#1}{#2}\!\big)}{\big(\!\binom{#1}{#2}\!\big)}}
\DeclareRobustCommand{\sbinom}{\genfrac{[}{]}{0pt}{}}
\DeclareRobustCommand{\lbinom}{\genfrac{\{}{\}}{0pt}{}}
\newcommand{\fallfac}[2]{{#1}^{\underline{#2}}}
\newcommand{\risefac}[2]{{#1}^{\overline{#2}}}

\newcommand{\defn}[1]{\textit{\textsc{[Def]\,}}\textbf{#1}\\[5pt]\indent}
\newcommand{\nin}{\noindent}
% macros
\newcommand{\QQ}{\mathbb Q}
\newcommand{\RR}{\mathbb R}
\newcommand{\NN}{\mathbb N}
\newcommand{\ZZ}{\mathbb Z}
\newcommand{\FF}{\mathbb F}
\newcommand{\CC}{\mathbb C}
\newcommand{\EE}{\mathbb E}
\DeclareMathOperator{\COM}{COM}
\DeclareMathOperator{\com}{com}
\DeclareMathOperator{\ord}{ord}
\DeclareMathOperator{\WCOM}{WCOM}
\DeclareMathOperator{\wcom}{wcom}
\newcommand{\sop}{\operatorname{sop}}

%Teoremas, Lemas, etc.

\theoremstyle{plain}
\newtheorem{teo}{Teorema}
\newtheorem{lem}[teo]{Lema}
\newtheorem{prop}{Proposici\'on}
\newtheorem{cor}[teo]{Corolario}
\newtheorem{cor*}{Corolario}
\theoremstyle{definition}
\newtheorem{defi}[teo]{Definici\'on}
\newtheorem{eje}[teo]{Ejemplo}
\newtheorem{ejeres}[teo]{Ejercicios resueltos}
\newtheorem{ejere}[teo]{Ejercicio resuelto}
\newtheorem{ejes}[teo]{Ejemplos}
\newtheorem{ejer}[teo]{Ejercicio}
\newtheorem{prob}[teo]{Problema}
\newtheorem{obs}[teo]{Observaci\'on}
\newtheoremstyle{Azul}
{\topsep}   % ABOVESPACE
{\topsep}   % BELOWSPACE
{\color{blue}}  % BODYFONT
{0pt}       % INDENT (empty value is the same as 0pt)
{\color{blue}\bfseries} % HEADFONT
{.}         % HEADPUNCT
{5pt plus 1pt minus 1pt}  % HEADSPACE. `plain` default: {5pt plus 1pt minus 1pt}
{}          % CUSTOM-HEAD-SPEC
\theoremstyle{Azul}
\newtheorem*{comm}{Comentario}
\newcommand{\commento}[1]{\noindent{\color{blue}#1}\vspace*{-3pt}}

% fin macros

\usepackage{fancyhdr}
\pagestyle{fancy}
\fancypagestyle{plain}{%
\fancyhf{}
\lhead{\footnotesize\itshape\bfseries\rightmark}
\rhead{\footnotesize\itshape\bfseries\leftmark}
}

% END_FOLD
\begin{document}
% BEGIN_FOLD encabezado
\setlength{\headheight}{14pt}
\fancyhead[L]{Facultad de Ciencias F\'isicas y Matem\'aticas}
\fancyhead[R]{Universidad de Chile}
\vspace*{-1.2 cm}
\begin{minipage}{0.6\textwidth}
	\begin{flushleft}
		\hspace*{-0.5cm}\textbf{MA4702. Programación Lineal Mixta 2020.}\\
		\hspace*{-0.5cm}\textbf{Profesor: José Soto.}\\
	\end{flushleft}
\end{minipage}
\begin{minipage}{0.36\textwidth}
	\begin{flushright}
		\includegraphics[scale=0.15]{fcfm}
	\end{flushright}
\end{minipage}
\bigskip
%Fin encabezado

% END_FOLD
\newif\ifsol
\soltrue
\solfalse

\begin{center}
  \LARGE \textbf{Tarea 1}.\\
\end{center}
\bigskip

\noindent\textbf{Fecha entrega}: Lunes 04/05 a las 12:00.\\
\textbf{Instrucciones:} Entregue los 4 problemas escaneados o en una foto de alta calidad, via ucursos.
Los problemas de la tarea se deben resolver de manera individual. Para asegurar trabajo personal se exige que las respuestas sean escritas a mano.\\

Hola
\noindent \textbf{¡No se aceptarán documentos tipeados o generados por computador!}\\
\textbf{¡No se aceptarán documentos tipeados o generados por computador!}\\
\textbf{¡No se aceptarán documentos tipeados o generados por computador!}\\
\textbf{¡No se aceptarán documentos tipeados o generados por computador!}\\

\begin{prob}
Escriba restricciones lineales mixtas (puede imponer integralidad o usar variables adicionales) para modelar las siguientes condiciones lógicas o algebraicas, dando una explicación breve de su solución.


\begin{enumerate}[(1)]
\item[(0)] (Ejemplo)  $x, y\in [3], x=3 \implies y\ge 2$\\
(Solución):   \begin{align*}
x\leq 3, y\leq 3, x\ge 1, y&\ge 1, x, y\in \ZZ\\
4x-y&\leq 10
\end{align*}
Explicación breve: Si $x\leq 2$, $4x$ es a lo más 8, y luego la desigualdad inferior se cumple para todo $y$.
Pero si $x=3$ entonces $4x$ es 12, y luego para que la desigualdad inferior se cumpla, $y$ debe ser al menos 2.

\item $x, y\in \{0,1\}$, $x=1 \implies y=1$.

\textbf{Solución:}\\
\begin{align*}
	x\leq y
\end{align*}

Está restricción impone que $y=1$ si $x=1$ y ninguna condición si $x=0$.
\item $x_i \in \{0,1\}, \forall i\in [m], y\in \{0,1\}$. Si más de $k$ variables $x_i$ son 1 entonces $y=1$.

\textbf{Solución:}\\
\begin{align*}
	\sum_{i=1}^{m}x_{i}-k\leq (m-k)y
\end{align*}
El lado izquierdo de esta restricción es mayor que 0 cuando más de $k$ variables $x_{i}$ son 1 y cuando esto courre $y$ es 1.

\item $x, y, z\in \{0,1\}, \text{si $x=1$, entonces ($y=1$ o $z=1$)}$

\textbf{Solución:}\\
\begin{align*}
	x\leq y+z
\end{align*}
Cuando $x=1$ se tiene que $y+z\geq1$, por ende, se tiene que alemnos $y$ o $z$ debe ser 1.

\item $x_i \in \{0,1\}, \forall i\in [m]$, $\prod_{i=1}^m x_i = 0$.

\textbf{Solución:}\\
\begin{align*}
	\sum_{i=1}^{m}x_{i}\leq m-1
\end{align*}
La restricción del enunciado impone que al menos un $x_i=0$ para que así el producto de todos sea 0, entonces, basta limitar que la suma de los $x_{i}$ sea menor estrico que $m$.

\item $x, y\in [n]$, $x$ e $y$ tienen la misma paridad.

\textbf{Solución:}\\
\begin{align*}
x+y=2z, \; z\in[n]
\end{align*}
Notar que la suma de dos valores con la misma paridad siempre es un número par, en cambio la suma de números con distinta paridad da un número impar.

\item $x\in [-M,M]$ (intervalo real), $y=|x|$.\\
 (Indicación, cree variables binarias $w_1, w_2$ que representen la expresión $w_1=1 \iff (x\geq 0 \wedge y=x)$; $w_2=1 \iff (x\leq 0 \wedge y=-x)$

\textbf{Solución:}\\
Creamos las variables binarias $w_{1}$ y $w_{2}$ para capturar el signo de $x$:
\begin{align}
x \leq M w_{1}, \; x\geq-Mw_{2}\\
0\leq y-x \leq 2M(1-w_{1})\\
0\leq y+x \leq 2M(1-w_{2})\\
w_{1}+w_{2}=1
\end{align}
Por (1) se tiene que $w_{1}=1$ si $x\geq0$ y $w_{2}=1$ si $x\leq0$, por (2) se tiene que $y=x$ cuando $w_{1}=1$ y por (3) se tiene que $y=-x$ cuando $w_{2}=1$. Por (4) se tiene la recíproca, si $w_{1}= 1$ se tiene que $w_{2}=0$, luego por (1) se tiene que $x\geq0$ y por (2) $y=x$, se obtiene resultado análogo si $w_{2}=1$.
\end{enumerate}
\end{prob}

\newpage

\begin{prob} El problema de la mochila fraccional se modela con el siguiente programa lineal puro:
$$\max\{v^Tx\colon x\in \RR^n, \sum_{i=1}^n s_ix_i \leq B; 0\leq x_i\leq 1, \forall i\in [n]\},$$
donde los tamaños $s_i>0, \forall i\in [n]$ y los valores son arbitrarios $v\in \RR^n_+$. Asuma además sin pérdida de generalidad que los objetos están ordenados de acuerdo a valor sobre tamaño, es decir $$\frac{v_1}{s_1}\geq \frac{v_2}{s_2}\geq \dots \geq \frac{v_n}{s_n}$$

Sea $h$ el índice más grande tal que los primeros $h$ objetos caben en la mochila: $\sum_{i=1}^h s_i \leq B$. Suponiendo que $h<n$, demuestre usando holgura complementaria que existe una solución óptima $x^*$ de este problema que satisface
\begin{align*}
x_i^*=1, \text{ para } i\in [h], \qquad x^*_{h+1}=\frac{B-\sum_{i=1}^h s_i}{s_{h+1}}, \qquad x^*_{i}=0, \text{ para }  i\geq h+2.
\end{align*}
\end{prob}

\textbf{Solución:}\\

Lo que se pide demostrar es que en el problema de la mochila fraccional en donde los productos se ordenan decrecientemente por utilidad marginal (por ejemplo, utilidad por $cm^{3}$) existe una solución óptima donde se van incorporando los productos de mayor utilidad marginal hasta que se llega al producto $h+1$ que no cabe completamente y entonces se llena el sobrante de la mochila con lo que quepa de este producto.

Antes de plantear las condiciones de holgura complementaria necesitamos obtener el dual del problema de la mochila:
\begin{align*}
	\max v^{T}x && \min 1^{T}y+Bz\\
	\sum_{i=1}^{n}s_{i}x_{i}&\leq B &   y_{i}+s_{i}z&\geq v_{i} \; \forall i\in[n]\\
	x_{i}&\leq 1 \;\forall i \in [n] &  y&\geq0 \;\forall i \in[n]\\
	x_{i}&\geq 0 \;\forall i \in[n] &  z&\geq0
\end{align*}

Recordar que un par solución primal-dual son óptimas ssi satisfacen holgura complementaria. Las condiciones de holgura complementaria son:

\begin{align}
	\left(\sum_{i=1}^{n}s_{i}x_{i}-B\right)z&=0\\
	(x_{i}-1)y_{i}&=0 \;\forall i \in[n]\\
	(y_{i}+s_{i}z-v_{i})x_{i}&=0 \;\forall i \in[n]
\end{align}
Por el enunciado sabemos que la capacidad restante de la mochila es $B-\sum_{i=1}^{h}s_{i}$ y que $x_{h+1}<1$, por ende, si fijamos $x_{h+1} = \frac{B-\sum_{i=1}^{h}s_{i}}{s_{h+1}}$ significa que $x_i=0 \; \forall i \in \{h+2, \ldots, n\}$, ya que llenaríamos el sobrante de la mochila con $h+1$. Por (5) se tiene que $y_{i}=0  \; \forall i \in \{h+1, \ldots, n\}$, como $x_{h+1}\in(0,1)$, reemplazando en (6) se tiene que $z=\frac{v_{h+1}}{s_{h+1}}$.\\

Veamos si existe una solución del dual que junto a esta solución del primal satisfacen las condiciones de holgura complementaria:

\begin{itemize}
	\item Como $z=\frac{v_{h+1}}{s_{h+1}}>0$ se tiene que cumplir que $\sum_{i=1}^{n}s_{i}x_{i}-B=0$, en efecto, $\sum_{i=1}^{n}s_{i}x_{i}=\sum_{i=1}^{h}s_{i}+ \left(\frac{B-\sum_{i=1}^{h}s_{i}}{s_{h+1}}\right)s_{h+1}=B$.
	\item Para $i \in \{1, \ldots, h\}$ se tiene que $x_{i}=1$, por ende $y_{i}+s_{i}z-v_{i}=0 \implies y_{i}=s_{i}\left(\frac{v_{i}}{s_{i}}-\frac{v_{h+1}}{s_{h+1}}\right)$, luego $y_{i}\geq0$ ya que $\frac{v_{i}}{s_{i}}\geq\frac{v_{j}}{s_{j}}$ si $i\leq j$.
	\item Se tiene que $x_{i}\in[0,1) \; \forall i \in \{h+1, \ldots, n\}$, esto implica que $y_{i}=0$, ahora basta demostrar que $y_{i}+s_{i}z\geq v_{i}$, lo que se cumple puesto que $\frac{v_{h+1}}{s_{h+1}}\geq \frac{v_{i}}{s_{i}} \; \forall i\geq h+1$.
\end{itemize}}


\newpage
\begin{prob} \textbf{Teorema de descomposición de flujo}. Sea $G=(V,E)$ un grafo dirigido y $s$, $t$ dos de sus vértices.
Considere los siguientes poliedros.
\begin{align*}
P_1=\{x\in \RR^E_{+}\colon
x(\delta^+(v))-x(\delta^- (v))&=0, \forall v\in V\setminus \{s,t\}, x(\delta^+(s))-x(\delta^- (s))\geq0\}.\\
P_2=\{x\in \RR^E_{+}\colon
x(\delta^+(v))-x(\delta^- (v))&=0, \forall v\in V\setminus \{s,t\},
x(\delta^+(s))-x(\delta^- (s))=1\}.\\
P_3=\{x\in \RR^E_{+}\colon
x(\delta^+(v))-x(\delta^- (v))&=0, \forall v\in V\setminus \{t\}\},
\end{align*}

Los elementos de $P_1$ se llaman $s$-$t$ flujos, los elementos de $P_2$ se llaman $s$-$t$ flujos de valor 1 y los elementos de $P_3$ se llaman $s$-$t$ flujos de valor 0, o circulaciones.

\begin{enumerate}[(a)]
\item Pruebe que $P_1$ y $P_3$ son conos, que $P_2\subseteq P_1$ y que $\text{rec}(P_2)=P_3$.

\textbf{Solución:}\\

En primer lugar se demostrará que $P_1$ y $P_3$ son conos, para esto notar que ambos se pueden escribir de la forma $Ax\leq0$, con $A_{ij}\in\{-1,0,1\}$, por ende, ambos son conos y de hecho son conos poliedrales.\\

Ahora se demostrará que $P_{2}\subseteq P_{1}:$\\

\begin{aligned}
	P_{2} & =\{x\in \RR^E_{+}\colon x(\delta^+(v))-x(\delta^- (v))=0, \forall v\in V\setminus \{s,t\},  x(\delta^+(s))-x(\delta^- (s))\geq0\}\cap \{x\in \RR^E_{+}:x(\delta^{+}(v))-x(\delta^{-}(s))=1\}\\
	& = P_{1} \cap \{x\in \RR^E_{+}:x(\delta^{+}(v))-x(\delta^{-}(s))=1\}
\end{aligned}\\

Por último, se demostrará que $rec(P_{2})=P_{3}$, en efecto, recordando que el cono de recesión de un poliedro de la forma $Ax\leq b$ es $Ax\leq0$ y $P_{3}=\hat{A}x\leq0$, luego basta notar que $A=\hat{A}$, puesto a que las restricciones de flujo están definidas sobre el mismo conjunto de nodos y las de no negatividad sobre el mismo conjunto de arcos.

\item Para todo $x\in P_{1}$, llame soporte de $x$ al conjunto de arcos $S_x=\{e\in E\colon x_e>0\}$. Demuestre que si $x$ no es el vector 0 entonces $S_x$ contiene a los arcos de algún $s$-$t$ camino $P$ o de un un ciclo $C$.

\textbf{Solución:}\\

Observación: Sea $v \in V\setminus\{s,t\}$, existe un arco entrando a $v$ en $S_x$ si y solo si existe un arco saliendo de $v$ en $S_x$. En efecto, como $x(\delta^-v)=x(\delta^+v)$, se tiene que $x(\delta^v)>0$ si y solo si $x(\delta^-v)>0$. La primera condición es equivalente a $|\delta^-(v) \cap S_x|>0$ y la segunda a $|\delta^+(v)\cap S_x|>0$.\\

Supongamos que $S_{x}$ no posee ciclos y demostremos que entonces debe tener un $s-t$ camino. Sea $W= \{e_{1}, e_{2}, \ldots, ..., e_{k}\}$  un paseo maximal en $S_{x}$ con $e_{i}=(a_{i},a_{i+1})$, un paseo es una secuencia de arcos distintos tal que la cabeza de $e_{i}$ es cola de $e_{i+1}$, es decir, si $e_{i} = (a_{i}, a_{i+1})$, entonces,  $e_{i+1}=(a_{i+1}, a_{i+2})$. Si el paseo no tiene ciclos, todos los $a_{i}$ son distintos y $W$ es un camino, si $a_{k+1}=v$ con $v \in V\setminus\{s,t\}$ entonces como $e_{k+1}$ entra a  v, debe existir un arco $e_{k+2}$ que sale de $v$ lo que nos permitiría  extender $W$ por la derecha a un camino más largo y no sería maximal, o bien cerrar un ciclo con algún vértice de $W$, es decir, $a_{k+1}=a_{1}$. Si $a_{k+1}=s$ entonces $x(\delta^-s)>0$, pero entonces como $x(\delta^+s)-x(\delta^-s)\geq0$ (\textbf{Observación:} si esto no se tenía en cuenta se llegaba a que $S_{x}$ además contenía los arcos de un t-s camino, respuesta igual de válida) se concluye que $x(\delta^+(s))>0$ y luego existe un arco $e_{k+2}$ que sale de $s$, lo que nos permitiría alargar el camino o cerrar un ciclo ($a_{k+1}=a_{1}=s$), por lo que $a_{k+1}=t$. Finalmente tenemos que $a_{1}$ solo puede ser $s$ o $t$ (pues si $a_{1}=v$ distinto de $s$ o $t$ y $e_1$ sale de $v$ debe haber un $e_{i}$ que entra a $v$, permitiendo extender $W$ o cerrar un ciclo), como no hay ciclos, $a_1$ debe ser $s$.

\item Llame $\mathcal{P}=\{P\subseteq E\colon \text{$P$ es un $s$-$t$ camino de $G$}\}$ y $\mathcal{C}=\{C\subseteq E\colon \text{$C$ es un ciclo de } $G$\}$. Pruebe usando la parte anterior que $P_1$ está cónicamente generado por $\{\chi^P\colon P\in \mathcal{P}\}\cup \{\chi^C\colon C\in \mathcal{C}\}$ donde $\chi^R\in \RR^E$ es el vector indicatriz de $R$, es decir $\chi^R_e$ es 1 si $e\in R$ y 0 si $e\not\in R$. \textbf{Indicación:} Pruebe que $x\in P_1$ tiene la combinación cónica pedida por inducción en el tamaño de $S_x$.

\textbf{Solución:}\\

Por demostrar que $P_{1}=cono(\{\chi^P\colon P\in \mathcal{P}\}\cup \{\chi^C\colon C\in \mathcal{C}\})$:


\textbf{$\supseteq$)}\\

Primero notar que $1_{\chi^{P}}$ y $1_{\chi^{C}}$ pertenecen a $P_{1}$, puesto a que son flujos unitarios que satisfacen las restricciones de conservación de flujo en los nodos $V \setminus \{s,t\}$. Luego la combinación cónica de sus elementos pertence $P_{1}$ ya que es cono.

\textbf{$\subseteq$)}\\

Notar que el poliedro $P_{1}$ se puede escribirse en la siguiente forma compacta:

\begin{align*}
	P_{1} = \left\{x\in R^{E}_{+}: Ax=0, A \in R^{|V|-2 \times |E|}, A_{ij} \in \{-1,0,1\} \; \forall i \in V\setminus\{s,t\}, j \in \{1, \ldots, |E|\}\right\}
\end{align*}

Si $x=0$ se tiene que $x$ pertenece al cono ya que todo cono contiene al $0$, si no por la parte anterior tenemos que $S_x$ contiene los arcos de un ciclo o un s-t camino, sea $\mathb{1}_{R^{1}}, R^{1}\in\mathcal{P}\cup\mathcal{C}$ un ciclo o s-t camino en $S_{x}$, sea $\lambda^{1} = \underset{e\in R^{1}}{min}\{x_{e}\}$ y $e^{1}=\underset{e\in R}{argmin}\{x_{e}\}$, definamos $\hat{x}=x-\lambda^{1}\mathb{1}_{R^{1}}$, notar que $\hat{x}\geq0$ y $\hat{x}=0$ solo si $x$ es un ciclo o un $s-t$ camino, en ese caso $x=\lambda^{1}\mathb{1}_{R^{1}}$ y terminaría la demostración. Notar que $\hat{x}\in P_{1}$, en efecto, $A\hat{x}=A(x-\lambda^{1}\mathb{1}_{R^{1}})=Ax-\lambda^{1}A\mathb{1}_{R^{1}}=0$, sea $S_{\hat{x}}=\{e\in E: \hat{x}_{e}>0\}$, notar que $|S_{\hat{x}}|\leq|S_{x}|-1$, debido a que el arco $e^{1}$ ahora tiene flujo 0, luego este procedimiento se puede repetir $k$ veces con $k\leq|S_{x}|$ hasta que $\hat{x}=0$, entonces, por inducción se tiene que $x=\sum_{i=1}^{k}\lambda^{i}\mathb{1}_{R^{i}}, \lambda\geq0$ donde $\mathb{1}_{R^{i}}$ es un ciclo o un $s-t$ camino $\forall i \in [k]$.


\item Deduzca la siguiente descomposición de Minkowski-Weyl del poliedro $P_2$:\\
$P_2=\text{conv}(\{\chi^P\colon P\in \mathcal{P}\}) + \text{cono}(\{\chi^C\colon C\in \mathcal{C}\})$.
\end{enumerate}

\textbf{Solución:}\\


\textbf{$\supseteq$)}\\

Sea $x=\underset{P \in \mathcal{P}}{\sum}\lambda_{P}\mathb{1}_{\chi^{P}}+\underset{C \in \mathcal{C}}{\sum}\theta_{C}\mathb{1}_{\chi^{C}}$, con $\lambda\geq0$, $\theta\geq0$ y $\underset{P\in\mathcal{P}}{\sum}\lambda_{P}=1$, luego $\forall v \in V\setminus\{s,t\}$ se tiene:

\begin{align*}
	x(\delta^{+}(v))-x(\delta^{-}(v)) &=\underset{e\in\delta^{+}(v)}{\sum}\underset{P \in \mathcal{P}}{\sum}\lambda_{P}\mathb{1}_{\chi^{P}_{e}}+\underset{C \in \mathcal{C}}{\sum}\theta_{C}\mathb{1}_{\chi^{C}_{e}}-\underset{e\in\delta^{-}(v)}{\sum}\underset{P \in \mathcal{P}}{\sum}\lambda_{P}\mathb{1}_{\chi^{P}_{e}}+\underset{C \in \mathcal{C}}{\sum}\theta_{C}\mathb{1}_{\chi^{C}_{e}}\\
	&=\underset{P \in \mathcal{P}}{\sum}\lambda_{P}\left(\underset{e\in\delta^{+}(v)}{\sum}\mathb{1}_{\chi^{P}_{e}}-\underset{e\in\delta^{-}(v)}{\sum}\mathb{1}_{\chi^{P}_{e}}\right)+\underset{C \in \mathcal{C}}{\sum}\theta_{C}\left(\underset{e\in\delta^{+}(v)}{\sum}\mathb{1}_{\chi^{C}_{e}}-\underset{e\in\delta^{-}(v)}{\sum}\mathb{1}_{\chi^{C}_{e}}\right)\\
	&=\underset{P \in \mathcal{P}}{\sum}\lambda_{P}0+\underset{C \in \mathcal{C}}{\sum}\theta_{C}0=0\\
\end{align*}

Para $s$ se tiene:

\begin{align*}
	x(\delta^{+}(s))-x(\delta^{-}(s)) &=\underset{P \in \mathcal{P}}{\sum}\lambda_{P}\left(\underset{e\in\delta^{+}(s)}{\sum}\mathb{1}_{\chi^{P}_{e}}-\underset{e\in\delta^{-}(s)}{\sum}\mathb{1}_{\chi^{P}_{e}}\right)+\underset{C \in \mathcal{C}}{\sum}\theta_{C}\left(\underset{e\in\delta^{+}(s)}{\sum}\mathb{1}_{\chi^{C}_{e}}-\underset{e\in\delta^{-}(s)}{\sum}\mathb{1}_{\chi^{C}_{e}}\right)\\
	&=\underset{P \in \mathcal{P}}{\sum}\lambda_{P}1+\underset{C \in \mathcal{C}}{\sum}\theta_{C}0=1\\
\end{align*}

Por tanto, $x\in P_{2}$.

\textbf{$\subseteq$)}\\

Por (a) se tiene que $P_{2}\subseteq P_{1}$ y como $P_{1}=cono(\{\chi^P\colon P\in \mathcal{P}\}\cup \{\chi^C\colon C\in \mathcal{C}\})$ se tiene que $x$ se puede escribir como una combinación cónica de $s-t$ caminos y ciclos, por tanto,  $\forall x\in P_{2}$ se tiene que existen $\lambda\geq0$, $\theta\geq0$ tal que $x=\underset{P \in \mathcal{P}}{\sum}\lambda_{P}\mathb{1}_{\chi^{P}}+\underset{C \in \mathcal{C}}{\sum}\theta_{C}\mathb{1}_{\chi^{C}}$, además como $x$ pertenece a $P_{2}$ debe satisfacer:

\begin{align*}
	1=x(\delta^{+}(s))-x(\delta^{-}(s)) & =\underset{P \in \mathcal{P}}{\sum}\lambda_{P}\left(\underset{e\in\delta^{+}(s)}{\sum}\mathb{1}_{\chi^{P}_{e}}-\underset{e\in\delta^{-}(s)}{\sum}\mathb{1}_{\chi^{P}_{e}}\right)+\underset{C \in \mathcal{C}}{\sum}\theta_{C}\left(\underset{e\in\delta^{+}(s)}{\sum}\mathb{1}_{\chi^{C}_{e}}-\underset{e\in\delta^{-}(s)}{\sum}\mathb{1}_{\chi^{C}_{e}}\right)\\
	&=\underset{P \in \mathcal{P}}{\sum}\lambda_{P}1+\underset{C \in \mathcal{C}}{\sum}\theta_{C}0=\underset{P \in \mathcal{P}}{\sum}\lambda_{P}
\end{align*}

Lo anterior implica que existen  $\lambda\geq0$ con $\underset{P\in\mathcal{P}}{\sum}\lambda_{P}=1$ y $\theta\geq0$ tal que $x=\underset{P \in \mathcal{P}}{\sum}\lambda_{P}\mathb{1}_{\chi^{P}}+\underset{C \in \mathcal{C}}{\sum}\theta_{C}\mathb{1}_{\chi^{C}}$.

\end{prob}

\newpage
\begin{prob}
Calcule el dual de los siguientes programas lineales puros (suponga $G=(V,E)$ es un grafo, que para todo $U\subseteq V$, $E(U)$ es el conjunto de aristas con ambos extremos en $U$, $\delta(U)$ es el conjunto de aristas con un extremo dentro de $U$ y uno fuera de $U$. )

\begin{alignat*}{5}
(P1): &\min\        & \sum_{e \in E} c_ex_e \\
&\text{s.a. } & x(\delta(v))&\geq 1 & \forall v\in V  \\
&                  & x_e                       &\geq 0 &\quad\forall e \in E
\intertext{}
(P2): &\max\        & \sum_{e \in E} w_ex_e \\
&\text{s.a. } & x(\delta(v))&\leq 1 & \forall v\in V  \\
&             & x(E(U))) &\leq \frac{|U|-1}{2} \quad & \forall U\subseteq V, |U| \text{ impar}\\
&             & x_e                       &\geq 0 &\quad\forall e \in E \\
(P3): &\max &\sum_{i=1}^I\sum_{j=1}^J\sum_{k=1}^K w_{ijk}x_{ijk}\\
&\text{s.a. }  &\sum_{j=1}^J\sum_{k=1}^K x_{ijk} &= a_{i} & \forall i\in [I].\\
&              &\sum_{i=1}^I\sum_{k=1}^K x_{ijk} &\leq b_{j} & \forall j\in [J].\\
&              &\sum_{i=1}^I\sum_{j=1}^J x_{ijk} &\geq c_{k} & \forall k\in [K].\\
&               &x_{ijk} &\geq 0 &\forall (i,j,k)\in [I]\times [J]\times [K].
\end{alignat*}


\textbf{Notas:} (P1) es el politopo de los edge-covers de $G$ cuando $G$ es bipartito, (P2) es el polítopo de los matchings de $G$, (P3) es la relajación de una variante de 3D-matching.


\textbf{Solución:}\\

\textbf{P1)}\\

Notar que un arco $(i,j)$ está presente en $\delta(i)$ y $\delta(j)$.
\begin{center}
\begin{aligned}
	\max \sum_{v\in V}y_{v}\\
	 y_{i}+y_{j} & \leq c_{ij} & \forall (i,j) \in E \\
	y_{v} &\geq 0 & \forall v \in V
\end{aligned}
\end{center}


\textbf{P2)}\\

Sea $S=\{1, \ldots, k\}$ enumeración de todos los $U\subseteq V, \; |U|$ impar y sea $\mathcal{U}=\{U_{1}, \ldots, U_{k}\}$ subconjunto con todos los subconjuntos de nodos de $V$ de cardinalidad impar.

\begin{center}
\begin{aligned}
	\min \sum_{v \in V}y_{v}+\sum_{s\in S}z_{s}\left( \frac{|\mathcal{U}(s)| -1}{2}\right)\\
	 y_{i}+y_{j}+\sum_{s\in S|(i,j)\in E(\mathcal{U}(s))}z_{s} & \geq w_{ij} & \forall (i,j) \in E \\
	y_{v} & \geq 0 & \forall v \in V\\
	z_{s} & \geq0 & \forall s \in S
\end{aligned}
\end{center}

\textbf{P3)}\\

\begin{center}
\begin{aligned}
	\min \sum_{i=1}^{I}a_{i}\lambda_{i} + \sum_{j=1}^{J}b_{j}\beta_{j}+ \sum_{k=1}^{K}c_{k}\gamma_{k}\\
	\lambda_{i}+\beta_{j}+\gamma_{k}&\geq w_{ijk} &\forall (i,j,k) \in [I]\times[J]\times[K]\\
	\lambda_{i} & \; \text{libre} &\forall i \in [I]\\
	\beta_{j}&\geq 0 &\forall j \in [J]\\
	\gamma_{k}&\leq0 &\forall k \in [K]
\end{aligned}
\end{center}

\end{prob}


\end{document}
