% BEGIN_FOLD preamble

\documentclass{article}
\usepackage[activeacute,spanish]{babel}
\usepackage{lmodern}
\usepackage[T1]{fontenc}
\usepackage{url,multirow}
\usepackage[centertableaux,smalltableaux]{ytableau}
\usepackage[utf8]{inputenc}
\usepackage{cancel, comment}
\usepackage[left=2cm,top=1.5cm,right=2cm, bottom=1.5cm,letterpaper, includeheadfoot]{geometry}
\usepackage{amssymb, amsmath, amsthm, mathtools}
\usepackage{graphicx}
\usepackage{hyperref}
\hypersetup{
	colorlinks,
	linkcolor={red!50!black},
	citecolor={blue!50!black},
	urlcolor={blue!80!black}
}

\usepackage[prependcaption,textsize=tiny,,textwidth=5cm]{todonotes}
\newcommand{\js}[1]{\todo[inline,linecolor=red,backgroundcolor=red!25,bordercolor=red]{jsoto. #1}}

\usepackage{paralist}
\usepackage{contour}


%%%Definiciones
\newcommand{\dis}{\displaystyle}
\newcommand{\IV}[1]{[\![#1]\!]} %Iverson

\newcommand{\E}{\mathcal{E}}
\newcommand{\V}{\mathcal{V}}

\def\multiset#1#2{\ensuremath{
		\mathchoice{\left(\kern-.3em\left(\genfrac{}{}{0pt}{}{#1}{#2}\right)\kern-.3em\right)}}
	{\big(\!\binom{#1}{#2}\!\big)}{\big(\!\binom{#1}{#2}\!\big)}{\big(\!\binom{#1}{#2}\!\big)}}
\DeclareRobustCommand{\sbinom}{\genfrac{[}{]}{0pt}{}}
\DeclareRobustCommand{\lbinom}{\genfrac{\{}{\}}{0pt}{}}
\newcommand{\fallfac}[2]{{#1}^{\underline{#2}}}
\newcommand{\risefac}[2]{{#1}^{\overline{#2}}}

\newcommand{\defn}[1]{\textit{\textsc{[Def]\,}}\textbf{#1}\\[5pt]\indent}
\newcommand{\nin}{\noindent}
% macros
\newcommand{\QQ}{\mathbb Q}
\newcommand{\RR}{\mathbb R}
\newcommand{\NN}{\mathbb N}
\newcommand{\ZZ}{\mathbb Z}
\newcommand{\FF}{\mathbb F}
\newcommand{\CC}{\mathbb C}
\newcommand{\EE}{\mathbb E}
\DeclareMathOperator{\COM}{COM}
\DeclareMathOperator{\com}{com}
\DeclareMathOperator{\ord}{ord}
\DeclareMathOperator{\WCOM}{WCOM}
\DeclareMathOperator{\wcom}{wcom}
\newcommand{\sop}{\operatorname{sop}}

%Teoremas, Lemas, etc.

\theoremstyle{plain}
\newtheorem{teo}{Teorema}
\newtheorem{lem}[teo]{Lema}
\newtheorem{prop}{Proposici\'on}
\newtheorem{cor}[teo]{Corolario}
\newtheorem{cor*}{Corolario}
\theoremstyle{definition}
\newtheorem{defi}[teo]{Definici\'on}
\newtheorem{eje}[teo]{Ejemplo}
\newtheorem{ejeres}[teo]{Ejercicios resueltos}
\newtheorem{ejere}[teo]{Ejercicio resuelto}
\newtheorem{ejes}[teo]{Ejemplos}
\newtheorem{ejer}[teo]{Ejercicio}
\newtheorem{prob}[teo]{Problema}
\newtheorem{obs}[teo]{Observaci\'on}
\newtheoremstyle{Azul}
{\topsep}   % ABOVESPACE
{\topsep}   % BELOWSPACE
{\color{blue}}  % BODYFONT
{0pt}       % INDENT (empty value is the same as 0pt)
{\color{blue}\bfseries} % HEADFONT
{.}         % HEADPUNCT
{5pt plus 1pt minus 1pt}  % HEADSPACE. `plain` default: {5pt plus 1pt minus 1pt}
{}          % CUSTOM-HEAD-SPEC
\theoremstyle{Azul}
\newtheorem*{comm}{Comentario}
\newcommand{\commento}[1]{\noindent{\color{blue}#1}\vspace*{-3pt}}

% fin macros

\usepackage{fancyhdr}
\pagestyle{fancy}
\fancypagestyle{plain}{%
\fancyhf{}
\lhead{\footnotesize\itshape\bfseries\rightmark}
\rhead{\footnotesize\itshape\bfseries\leftmark}
}

% END_FOLD
\begin{document}
% BEGIN_FOLD encabezado
\setlength{\headheight}{14pt}
\fancyhead[L]{Facultad de Ciencias F\'isicas y Matem\'aticas}
\fancyhead[R]{Universidad de Chile}
\vspace*{-1.2 cm}
\begin{minipage}{0.6\textwidth}
	\begin{flushleft}
		\hspace*{-0.5cm}\textbf{MA4702. Programación Lineal Mixta 2020.}\\
		\hspace*{-0.5cm}\textbf{Profesor: José Soto.}\\
	\end{flushleft}
\end{minipage}
\begin{minipage}{0.36\textwidth}
	\begin{flushright}
		\includegraphics[scale=0.15]{fcfm}
	\end{flushright}
\end{minipage}
\bigskip
%Fin encabezado

% END_FOLD
\newif\ifsol
\soltrue
\solfalse

\begin{center}
  \LARGE \textbf{Tarea 1}.\\
\end{center}
\bigskip

\noindent\textbf{Fecha entrega}: Lunes 04/05 a las 12:00.\\
\textbf{Instrucciones:} Entregue los 4 problemas escaneados o en una foto de alta calidad, via ucursos.
Los problemas de la tarea se deben resolver de manera individual. Para asegurar trabajo personal se exige que las respuestas sean escritas a mano.\\


\noindent \textbf{¡No se aceptarán documentos tipeados o generados por computador!}\\
\textbf{¡No se aceptarán documentos tipeados o generados por computador!}\\
\textbf{¡No se aceptarán documentos tipeados o generados por computador!}\\
\textbf{¡No se aceptarán documentos tipeados o generados por computador!}\\

\begin{prob}
Escriba restricciones lineales mixtas (puede imponer integralidad o usar variables adicionales) para modelar las siguientes condiciones lógicas o algebraicas, dando una explicación breve de su solución.


\begin{enumerate}[(1)]
\item[(0)] (Ejemplo)  $x, y\in [3], x=3 \implies y\ge 2$\\
(Solución):   \begin{align*}
x\leq 3, y\leq 3, x\ge 1, y&\ge 1, x, y\in \ZZ\\
4x-y&\leq 10
\end{align*}
Explicación breve: Si $x\leq 2$, $4x$ es a lo más 8, y luego la desigualdad inferior se cumple para todo $y$.
Pero si $x=3$ entonces $4x$ es 12, y luego para que la desigualdad inferior se cumpla, $y$ debe ser al menos 2.

\item $x, y\in \{0,1\}$, $x=1 \implies y=1$.

\item $x_i \in \{0,1\}, \forall i\in [m], y\in \{0,1\}$. Si más de $k$ variables $x_i$ son 1 entonces $y=1$.

\item $x, y, z\in \{0,1\}, \text{si $x=1$, entonces ($y=1$ o $z=1$)}$

\item $x_i \in \{0,1\}, \forall i\in [m]$, $\prod_{i=1}^m x_i = 0$.

\item $x, y\in [n]$, $x$ e $y$ tienen la misma paridad.

\item $x\in [-M,M]$ (intervalo real), $y=|x|$.\\
 (Indicación, cree variables binarias $w_1, w_2$ que representen la expresión $w_1=1 \iff (x\geq 0 \wedge y=x)$; $w_2=1 \iff (x\leq 0 \wedge y=-x)$)

\end{enumerate}
\end{prob}
\begin{prob} El problema de la mochila fraccional se modela con el siguiente programa lineal puro:
$$\max\{v^Tx\colon x\in \RR^n, \sum_{i=1}^n s_ix_i \leq B; 0\leq x_i\leq 1, \forall i\in [n]\},$$
donde los tamaños $s_i>0, \forall i\in [n]$ y los valores son arbitrarios $v\in \RR^n_+$. Asuma además sin pérdida de generalidad que los objetos están ordenados de acuerdo a valor sobre tamaño, es decir $$\frac{v_1}{s_1}\geq \frac{v_2}{s_2}\geq \dots \geq \frac{v_n}{s_n}$$

Sea $h$ el índice más grande tal que los primeros $h$ objetos caben en la mochila: $\sum_{i=1}^h s_i \leq B$. Suponiendo que $h<n$, demuestre usando holgura complementaria que existe una solución óptima $x^*$ de este problema que satisface
\begin{align*}
x_i^*=1, \text{ para } i\in [h], \qquad x^*_{h+1}=\frac{B-\sum_{i=1}^h s_i}{s_{h+1}}, \qquad x^*_{i}=0, \text{ para }  i\geq h+2.
\end{align*}
\end{prob}
\begin{prob} \textbf{Teorema de descomposición de flujo}. Sea $G=(V,E)$ un grafo dirigido y $s$, $t$ dos de sus vértices.
Considere los siguientes poliedros.
\begin{align*}
P_1=\{x\in \RR^E_{+}\colon
x(\delta^+(v))-x(\delta^- (v))&=0, \forall v\in V\setminus \{s,t\}, x(\delta^+(s))-x(\delta^- (s))\geq0\}.\\
P_2=\{x\in \RR^E_{+}\colon
x(\delta^+(v))-x(\delta^- (v))&=0, \forall v\in V\setminus \{s,t\},
x(\delta^+(s))-x(\delta^- (s))=1\}.\\
P_3=\{x\in \RR^E_{+}\colon
x(\delta^+(v))-x(\delta^- (v))&=0, \forall v\in V\setminus \{t\}\},
\end{align*}

Los elementos de $P_1$ se llaman $s$-$t$ flujos, los elementos de $P_2$ se llaman $s$-$t$ flujos de valor 1 y los elementos de $P_3$ se llaman $s$-$t$ flujos de valor 0, o circulaciones.

\begin{enumerate}[(a)]
\item Pruebe que $P_1$ y $P_3$ son conos, que $P_2\subseteq P_1$ y que $\text{rec}(P_2)=P_3$.
\item Para todo $x\in P_{1}$, llame soporte de $x$ al conjunto de arcos $S_x=\{e\in E\colon x_e>0\}$. Demuestre que si $x$ no es el vector 0 entonces $S_x$ contiene a los arcos de algún $s$-$t$ camino $P$ o de un un ciclo $C$.
\item Llame $\mathcal{P}=\{P\subseteq E\colon \text{$P$ es un $s$-$t$ camino de $G$}\}$ y $\mathcal{C}=\{C\subseteq E\colon \text{$C$ es un ciclo de } $G$\}$. Pruebe usando la parte anterior que $P_1$ está cónicamente generado por $\{\chi^P\colon P\in \mathcal{P}\}\cup \{\chi^C\colon C\in \mathcal{C}\}$ donde $\chi^R\in \RR^E$ es el vector indicatriz de $R$, es decir $\chi^R_e$ es 1 si $e\in R$ y 0 si $e\not\in R$. \textbf{Indicación:} Pruebe que $x\in P_1$ tiene la combinación cónica pedida por inducción en el tamaño de $S_x$.
\item Deduzca la siguiente descomposición de Minkowski-Weyl del poliedro $P_2$:\\
$P_2=\text{conv}(\{\chi^P\colon P\in \mathcal{P}\}) + \text{cono}(\{\chi^C\colon C\in \mathcal{C}\})$.
\end{enumerate}
\end{prob}

\begin{prob}
Calcule el dual de los siguientes programas lineales puros (suponga $G=(V,E)$ es un grafo, que para todo $U\subseteq V$, $E(U)$ es el conjunto de aristas con ambos extremos en $U$, $\delta(U)$ es el conjunto de aristas con un extremo dentro de $U$ y uno fuera de $U$. )

\begin{alignat*}{5}
(P1): &\min\        & \sum_{e \in E} c_ex_e \\
&\text{s.a. } & x(\delta(v))&\geq 1 & \forall v\in V  \\
&                  & x_e                       &\geq 0 &\quad\forall e \in E
\intertext{}
(P2): &\max\        & \sum_{e \in E} w_ex_e \\
&\text{s.a. } & x(\delta(v))&\leq 1 & \forall v\in V  \\
&             & x(E(U))) &\leq \frac{|U|-1}{2} \quad & \forall U\subseteq V, |U| \text{ impar}\\
&             & x_e                       &\geq 0 &\quad\forall e \in E \\
(P3): &\max &\sum_{i=1}^I\sum_{j=1}^J\sum_{k=1}^K w_{ijk}x_{ijk}\\
&\text{s.a. }  &\sum_{j=1}^J\sum_{k=1}^K x_{ijk} &= a_{i} & \forall i\in [I].\\
&              &\sum_{i=1}^I\sum_{k=1}^K x_{ijk} &\leq b_{j} & \forall j\in [J].\\
&              &\sum_{i=1}^I\sum_{j=1}^J x_{ijk} &\geq c_{k} & \forall k\in [K].\\
&               &x_{ijk} &\geq 0 &\forall (i,j,k)\in [I]\times [J]\times [K].
\end{alignat*}


\textbf{Notas:} (P1) es el politopo de los edge-covers de $G$ cuando $G$ es bipartito, (P2) es el polítopo de los matchings de $G$, (P3) es la relajación de una variante de 3D-matching.
\end{prob}


	\end{document}
