% BEGIN_FOLD preamble

\documentclass{article}
\usepackage[activeacute,spanish]{babel}
\usepackage{lmodern}
\usepackage[T1]{fontenc}
\usepackage{url,multirow}
\usepackage[centertableaux,smalltableaux]{ytableau}
\usepackage[utf8]{inputenc}
\usepackage{cancel, comment}
\usepackage[left=2cm,top=1.5cm,right=2cm, bottom=1.5cm,letterpaper, includeheadfoot]{geometry}
\usepackage{amssymb, amsmath, amsthm, mathtools}
\usepackage{graphicx}
\usepackage{hyperref}
\hypersetup{
	colorlinks,
	linkcolor={red!50!black},
	citecolor={blue!50!black},
	urlcolor={blue!80!black}
}

\usepackage[prependcaption,textsize=tiny,,textwidth=5cm]{todonotes}
\newcommand{\js}[1]{\todo[inline,linecolor=red,backgroundcolor=red!25,bordercolor=red]{jsoto. #1}}

\usepackage{paralist}
\usepackage{contour}
\usepackage{algorithm, algorithmic}

%%%Definiciones
\newcommand{\dis}{\displaystyle}
\newcommand{\IV}[1]{[\![#1]\!]} %Iverson

\newcommand{\E}{\mathcal{E}}
\newcommand{\V}{\mathcal{V}}

\def\multiset#1#2{\ensuremath{
		\mathchoice{\left(\kern-.3em\left(\genfrac{}{}{0pt}{}{#1}{#2}\right)\kern-.3em\right)}}
	{\big(\!\binom{#1}{#2}\!\big)}{\big(\!\binom{#1}{#2}\!\big)}{\big(\!\binom{#1}{#2}\!\big)}}
\DeclareRobustCommand{\sbinom}{\genfrac{[}{]}{0pt}{}}
\DeclareRobustCommand{\lbinom}{\genfrac{\{}{\}}{0pt}{}}
\newcommand{\fallfac}[2]{{#1}^{\underline{#2}}}
\newcommand{\risefac}[2]{{#1}^{\overline{#2}}}

\newcommand{\defn}[1]{\textit{\textsc{[Def]\,}}\textbf{#1}\\[5pt]\indent}
\newcommand{\nin}{\noindent}
% macros
\newcommand{\QQ}{\mathbb Q}
\newcommand{\RR}{\mathbb R}
\newcommand{\NN}{\mathbb N}
\newcommand{\ZZ}{\mathbb Z}
\newcommand{\FF}{\mathbb F}
\newcommand{\CC}{\mathbb C}
\newcommand{\EE}{\mathbb E}
\DeclareMathOperator{\COM}{COM}
\DeclareMathOperator{\com}{com}
\DeclareMathOperator{\ord}{ord}
\DeclareMathOperator{\DFJ}{DFJ}
\DeclareMathOperator{\EUL}{EUL}
\DeclareMathOperator{\WCOM}{WCOM}
\DeclareMathOperator{\wcom}{wcom}
\newcommand{\sop}{\operatorname{sop}}

%Teoremas, Lemas, etc.

\theoremstyle{plain}
\newtheorem{teo}{Teorema}
\newtheorem{lem}[teo]{Lema}
\newtheorem{prop}{Proposici\'on}
\newtheorem{cor}[teo]{Corolario}
\newtheorem{cor*}{Corolario}
\theoremstyle{definition}
\newtheorem{defi}[teo]{Definici\'on}
\newtheorem{eje}[teo]{Ejemplo}
\newtheorem{ejeres}[teo]{Ejercicios resueltos}
\newtheorem{ejere}[teo]{Ejercicio resuelto}
\newtheorem{ejes}[teo]{Ejemplos}
\newtheorem{ejer}[teo]{Ejercicio}
\newtheorem{prob}[teo]{Problema}
\newtheorem{obs}[teo]{Observaci\'on}
\newtheoremstyle{Azul}
{\topsep}   % ABOVESPACE
{\topsep}   % BELOWSPACE
{\color{blue}}  % BODYFONT
{0pt}       % INDENT (empty value is the same as 0pt)
{\color{blue}\bfseries} % HEADFONT
{.}         % HEADPUNCT
{5pt plus 1pt minus 1pt}  % HEADSPACE. `plain` default: {5pt plus 1pt minus 1pt}
{}          % CUSTOM-HEAD-SPEC
\theoremstyle{Azul}
\newtheorem*{comm}{Comentario}
\newcommand{\commento}[1]{\noindent{\color{blue}#1}\vspace*{-3pt}}

% fin macros

\usepackage{fancyhdr}
\pagestyle{fancy}
\fancypagestyle{plain}{%
\fancyhf{}
\lhead{\footnotesize\itshape\bfseries\rightmark}
\rhead{\footnotesize\itshape\bfseries\leftmark}
}

% END_FOLD
\begin{document}
% BEGIN_FOLD encabezado
\setlength{\headheight}{14pt}
\fancyhead[L]{Facultad de Ciencias F\'isicas y Matem\'aticas}
\fancyhead[R]{Universidad de Chile}
\vspace*{-1.2 cm}
\begin{minipage}{0.6\textwidth}
	\begin{flushleft}
		\hspace*{-0.5cm}\textbf{MA4702. Programación Lineal Mixta 2020.}\\
		\hspace*{-0.5cm}\textbf{Profesor: José Soto.}\\
	\end{flushleft}
\end{minipage}
\begin{minipage}{0.36\textwidth}
	\begin{flushright}
		\includegraphics[scale=0.15]{fcfm.pdf}
	\end{flushright}
\end{minipage}
\bigskip
%Fin encabezado

% END_FOLD
\newif\ifsol
\soltrue
\solfalse

\begin{center}
  \LARGE \textbf{Tarea 3}.\\
\end{center}
\bigskip

\noindent\textbf{Fecha entrega}: Lunes 15 de Junio, 23:59. Por u-cursos.
\subsection*{Instrucciones:} 
\begin{enumerate}
\item \textbf{Extensión máxima}: Entregue su tarea en a lo más \textbf{6 planas}.
    \item \textbf{Formato:} La tarea debe entregarse en formato pdf, con fondo de un solo color (blanco de preferencia), letra legible en manuscrito y clara. (No se aceptarán documentos tipeados o generados por computador, pero si tiene alguna manera de escribir en manuscrito directamente de manera digital lo puede hacer).
    Si desarrolla su tarea en papel, entréguelo escaneados o en fotos de alta calidad, via ucursos.    \item \textbf{Tiempo de dedicación:} El tiempo estimado de \emph{desarrollo} de la tarea es de 2.5 horas de dedicación. Esto no considera el tiempo de estudio previo, el tiempo dedicado en asistir a cátedras y auxiliares, ni el tiempo para \emph{ponerse al día}. Tendrá un plazo de 7 días para entregarlo. No espere hasta el último momento para escanear o fotografiar adecuadamente su tarea y cambiarla al formato solicitado (pdf). Entregue con suficiente anticipación a la hora límite.
    \item \textbf{Revisión:} Se podrá descontar hasta 1 punto en la nota final por falta de formato o extensión.
\item Esta tarea está pensada para ser hecha en forma individual.
    \end{enumerate}
\subsection*{Ejercicios:}
\begin{enumerate}[(a)]
\item {} [15 puntos] Sean $S,T\subseteq \RR^n$. Pruebe que $\text{conv}(S + T)=\text{conv}(S)+\text{conv}(T)$.
\item {} [15 puntos] Sean $P$, $Q$ polítopos con vértices $V(P)$ y $V(Q)$ respectivamente. Demuestre que
$R=\text{conv}(P\cup Q)$ es polítopo y que si $V(R)$ son los vértices de $R$ entonces $V(R)\subseteq V(P)\cup V(Q)$.
\item {} [30 puntos] Considere la variante del cutting stock problem en el cual cada cliente $i$ desea \textbf{a lo más} $b_i$ rollos de ancho $w_i$ y está dispuesto a pagar $g_i$ pesos por cada rollo de dicho ancho recibido (y no recibirá más de $b_i$). Además, la papelera incurre en un costo fijo de $s$ pesos por cada tronco de ancho $W$ usado, y dispone de no más de $N$ rollos. La papelera desea maximizar su utilidad definida como el ingreso recibido por los rollos vendidos menos el costo de los troncos usados. Determine:\\[2pt]


(c1) Modele el problema como un programa lineal entero (PE) que solo use variables $x_p$ para cada posible patrón $p\in \mathcal{P}$ (use la misma definición de patrón usada en clases y laboratorio).

\textbf{Indicación:} Defina explícitamente $g_p$ como el ingreso que le reporta vender los rollos de cierto patrón $p$. Este valor es una constante para (PE).\\[2pt]

(c2) Para $Q\subseteq \mathcal{P}$, el master problem MP$(Q)$ asociado y su dual DUAL-MP$(Q)$.

\textbf{Indicación:} Recuerde que MP$(Q)$ se obtiene tomando la relajación lineal de (PE) y eliminando (fijando a cero) todas las variables $x_p$ para $p$ fuera de $Q$. Su formulación MP$(Q)$ solo debe incluir variables $x_p$ con $p\in Q$ (no debe hacer mención al resto de los $x_p$).\\[2pt]

(c3) Para una solución dual factible $q$ de DUAL-MP$(Q)$ dada, el pricing problem asociado. Escriba este pricing problem como un programa lineal entero e interprételo como un problema de mochila.

\textbf{Indicación:} Recuerde que el pricing problem consiste en determinar cual es el índice (columna) asociada a la restricción de DUAL-MP$(\mathcal{P})$ que más viola $q$ (la menos satisfecha).
\end{enumerate}




	\end{document}
	


