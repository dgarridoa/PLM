% BEGIN_FOLD preamble

\documentclass{article}
\usepackage[activeacute,spanish]{babel}
\usepackage{lmodern}
\usepackage[T1]{fontenc}
\usepackage{url,multirow}
\usepackage[centertableaux,smalltableaux]{ytableau}
\usepackage[utf8]{inputenc}
\usepackage{cancel, comment}
\usepackage[left=2cm,top=1.5cm,right=2cm, bottom=1.5cm,letterpaper, includeheadfoot]{geometry}
\usepackage{amssymb, amsmath, amsthm, mathtools}
\usepackage{graphicx}
\usepackage{hyperref}
\hypersetup{
	colorlinks,
	linkcolor={red!50!black},
	citecolor={blue!50!black},
	urlcolor={blue!80!black}
}

\usepackage[prependcaption,textsize=tiny,,textwidth=5cm]{todonotes}
\newcommand{\js}[1]{\todo[inline,linecolor=red,backgroundcolor=red!25,bordercolor=red]{jsoto. #1}}

\usepackage{paralist}
\usepackage{contour}
\usepackage{algorithm, algorithmic}

%%%Definiciones
\newcommand{\dis}{\displaystyle}
\newcommand{\IV}[1]{[\![#1]\!]} %Iverson

\newcommand{\E}{\mathcal{E}}
\newcommand{\V}{\mathcal{V}}

\def\multiset#1#2{\ensuremath{
		\mathchoice{\left(\kern-.3em\left(\genfrac{}{}{0pt}{}{#1}{#2}\right)\kern-.3em\right)}}
	{\big(\!\binom{#1}{#2}\!\big)}{\big(\!\binom{#1}{#2}\!\big)}{\big(\!\binom{#1}{#2}\!\big)}}
\DeclareRobustCommand{\sbinom}{\genfrac{[}{]}{0pt}{}}
\DeclareRobustCommand{\lbinom}{\genfrac{\{}{\}}{0pt}{}}
\newcommand{\fallfac}[2]{{#1}^{\underline{#2}}}
\newcommand{\risefac}[2]{{#1}^{\overline{#2}}}

\newcommand{\defn}[1]{\textit{\textsc{[Def]\,}}\textbf{#1}\\[5pt]\indent}
\newcommand{\nin}{\noindent}
% macros
\newcommand{\QQ}{\mathbb Q}
\newcommand{\RR}{\mathbb R}
\newcommand{\NN}{\mathbb N}
\newcommand{\ZZ}{\mathbb Z}
\newcommand{\FF}{\mathbb F}
\newcommand{\CC}{\mathbb C}
\newcommand{\EE}{\mathbb E}
\DeclareMathOperator{\COM}{COM}
\DeclareMathOperator{\com}{com}
\DeclareMathOperator{\ord}{ord}
\DeclareMathOperator{\DFJ}{DFJ}
\DeclareMathOperator{\EUL}{EUL}
\DeclareMathOperator{\WCOM}{WCOM}
\DeclareMathOperator{\wcom}{wcom}
\newcommand{\sop}{\operatorname{sop}}

%Teoremas, Lemas, etc.

\theoremstyle{plain}
\newtheorem{teo}{Teorema}
\newtheorem{lem}[teo]{Lema}
\newtheorem{prop}{Proposici\'on}
\newtheorem{cor}[teo]{Corolario}
\newtheorem{cor*}{Corolario}
\theoremstyle{definition}
\newtheorem{defi}[teo]{Definici\'on}
\newtheorem{eje}[teo]{Ejemplo}
\newtheorem{ejeres}[teo]{Ejercicios resueltos}
\newtheorem{ejere}[teo]{Ejercicio resuelto}
\newtheorem{ejes}[teo]{Ejemplos}
\newtheorem{ejer}[teo]{Ejercicio}
\newtheorem{prob}[teo]{Problema}
\newtheorem{obs}[teo]{Observaci\'on}
\newtheoremstyle{Azul}
{\topsep}   % ABOVESPACE
{\topsep}   % BELOWSPACE
{\color{blue}}  % BODYFONT
{0pt}       % INDENT (empty value is the same as 0pt)
{\color{blue}\bfseries} % HEADFONT
{.}         % HEADPUNCT
{5pt plus 1pt minus 1pt}  % HEADSPACE. `plain` default: {5pt plus 1pt minus 1pt}
{}          % CUSTOM-HEAD-SPEC
\theoremstyle{Azul}
\newtheorem*{comm}{Comentario}
\newcommand{\commento}[1]{\noindent{\color{blue}#1}\vspace*{-3pt}}

% fin macros

\usepackage{fancyhdr}
\pagestyle{fancy}
\fancypagestyle{plain}{%
\fancyhf{}
\lhead{\footnotesize\itshape\bfseries\rightmark}
\rhead{\footnotesize\itshape\bfseries\leftmark}
}

% END_FOLD
\begin{document}
% BEGIN_FOLD encabezado
\setlength{\headheight}{14pt}
\fancyhead[L]{Facultad de Ciencias F\'isicas y Matem\'aticas}
\fancyhead[R]{Universidad de Chile}
\vspace*{-1.2 cm}
\begin{minipage}{0.6\textwidth}
	\begin{flushleft}
		\hspace*{-0.5cm}\textbf{MA4702. Programación Lineal Mixta 2020.}\\
		\hspace*{-0.5cm}\textbf{Profesor: José Soto.}\\
	\end{flushleft}
\end{minipage}
\begin{minipage}{0.36\textwidth}
	\begin{flushright}
		\includegraphics[scale=0.15]{fcfm.pdf}
	\end{flushright}
\end{minipage}
\bigskip
%Fin encabezado

% END_FOLD
\newif\ifsol
\soltrue
\solfalse

\begin{center}
  \LARGE \textbf{Tarea 4}.\\
\end{center}
\bigskip

\noindent\textbf{Fecha entrega}: Lunes 13 de Julio, 23:59. Por u-cursos.
\subsection*{Instrucciones:} 
\begin{enumerate}
\item \textbf{Extensión máxima}: Entregue su tarea en a lo más \textbf{6 planas}.
    \item \textbf{Formato:} La tarea debe entregarse en formato pdf, con fondo de un solo color (blanco de preferencia), letra legible en manuscrito y clara. (No se aceptarán documentos tipeados o generados por computador, pero si tiene alguna manera de escribir en manuscrito directamente de manera digital lo puede hacer).
    Si desarrolla su tarea en papel, entréguelo escaneados o en fotos de alta calidad, via ucursos.    \item \textbf{Tiempo de dedicación:} El tiempo estimado de \emph{desarrollo} de la tarea es de 3 horas de dedicación. Esto no considera el tiempo de estudio previo, el tiempo dedicado en asistir a cátedras y auxiliares, ni el tiempo para \emph{ponerse al día}. \textbf{En particular, para la última parte se recomienda repasar la clase 19 del curso}.
     Tendrá un plazo de 7 días para entregarlo. No espere hasta el último momento para escanear o fotografiar adecuadamente su tarea y cambiarla al formato solicitado (pdf). Entregue con suficiente anticipación a la hora límite.
    \item \textbf{Revisión:} Se podrá descontar hasta 1 punto en la nota final por falta de formato o extensión.
\item Esta tarea está pensada para ser hecha en forma individual.
    \end{enumerate}
    
  \subsection*{Definiciones para esta tarea}
Sea $N$ un conjunto finito. Decimos que dos conjuntos $A, B \in 2^N$ son \emph{intersectantes} si los conjuntos $A\setminus B$, $B\setminus A$ y $A\cap B$ son todos no vacíos. Notamos que cuando dos conjuntos no son intersectantes entonces debemos tener que $A$ y $B$ son disjuntos, o bien uno contiene al otro.

Una familia $\mathcal{L}\subseteq 2^N$ de subconjuntos de $N$ se dice \textbf{familia laminar sobre $N$} si no posee pares de conjuntos intersectantes, es decir, si para todo $A, B\in \mathcal{L}$.
\begin{align*}
(A\subseteq B) \text{ o } (B\subseteq A) \text{ ó } 
(A\cap B = \emptyset)
\end{align*}


\subsection*{Ejercicios:}
\noindent \textbf{Parte I (30 puntos).} \textbf{Elija, resuelva y entregue 3 ejercicios de esta parte $I$.} Si entrega más que 3 se calificarán los 3 peores ejercicios entregados.
\begin{enumerate}[(a)]
\item Demuestre la siguiente dirección del Teorema de Ghoulia-Houri: 
Si $A\in \{-1,0,1\}^{m\times n}$ es una matriz totalmente unimodular, entonces para todo $J\subseteq [n]$ existe una partición $J_1, J_2$ de $J$ tal que
$$\sum_{j\in J_1}a_{ij}-\sum_{j\in J_2}a_{ij} \in \{0,1,-1\}, \text{ para todo } i \in [m].$$
\textbf{Indicación:} Sea $B=A^J$. Demuestre que el poliedro $P=\{x\in \RR^J\colon \lfloor \frac12 B 1 \rfloor \leq Bx\leq \lceil \frac12 B 1\rceil, 0\leq x \leq 1\}$ es no vacío, y concluya que tiene un punto  $y\in \{0,1\}^J$ integral. Use ese punto $y$ para determinar la partición pedida (el vector $y$ tiene solo 2 tipos de valores).
\item Sea $\mathcal{I}$ una familia de intervalos cerrados, todos subconjuntos de $[0,n]$ y cada uno con sus extremos enteros. El \emph{minimum hitting set problem} consiste en encontrar el conjunto $H\subseteq [0,n]\cap \ZZ$ de menor cardinalidad tal que cada intervalo $J\in \mathcal{I}$ contiene al menos un punto de $H$. Encuentre un programa lineal puro que permita resolver este problema. \textbf{Indicación:} Demuestre primero que el programa lineal entero natural, tiene conjunto factible integral.
\item Sea $A\in \RR^{m\times n}$, con $m\leq n$. Demuestre que $A$ es totalmente unimodular si y solo si $[A | I]$ es unimodular.
\item Sea $G=(V,E)$ un digrafo, y sean $s$, $t$ dos nodos distintos de $V$. Encuentre un programa lineal puro que permita resolver el problema del $s$-$t$ camino de largo (número de aristas) mínimo.\\ \textbf{Indicación:} Escriba primero este problema como un flujo de costo mínimo con restricciones adicionales y argumente que esta versión del problema es integral.
\item Sea $\mathcal{L}$ una familia laminar sobre un conjunto $V$ y sea $G=(V,E)$ un digrafo. Se define la matriz de corte de entrada de $G$ respecto a $\mathcal{L}$ como la matriz $M\in \{0,1\}^{E,\mathcal{L}}$ tal que $M_{e, A}=1$ si y solo si $e\in \delta^-(A)$. 
Demuestre que la matriz $M$ es totalmente unimodular.

\textbf{Indicación:} Puede usar sin demostrar que cualquier subfamilia $\mathcal{F}$ de $\mathcal{L}$ también es laminar. Apóyese en un dibujo ¿puede ponerle signos a los conjuntos adecuados de modo que el número de conjuntos positivos que cada arco $e$ entre difiera poco del número de conjuntos negativos con esta propiedad? Use Ghoulia-Houri.
\item Sea $G=(V,E)$ un digrafo. Demuestre que para todo par de conjuntos $A, B\subseteq V$ y todo arco $e=(s,t)\in E$, se tiene la siguiente desigualdad
$$\IV{e\in \delta^-(A\cup B)}+\IV{e\in \delta^-(A\cap B)}-\IV{e\in \delta^-(A)}-\IV{e\in \delta^-(B)} \leq 0$$
donde $\IV{p}=1$ si $p$ es verdadero, y $\IV{p}=0$ en otro caso.
Concluya que para toda función $w\colon E \to \RR_+$, la función $f\colon 2^V\to \RR$ dada por $f(X)=w(\delta^-(X)) = \sum_{e\in \delta^-(X)}w_e$ es submodular.
\end{enumerate}

\noindent \textbf{Parte II (30 puntos).} \\
\textbf{Nota}: para hacer esta parte recomiendo basarse en las clases 19/20 del curso.\\

Sea $G=(V,E)$ un digrafo y sea $r\in V$ un nodo raiz. Un $r$-conector es cualquier subconjunto $F$ de arcos de $E$ tal que para cada nodo $v\in V$ existe un $r$-$v$ camino dirigido en $F$. Supongamos para evitar conjuntos vacíos que el conjunto $E$ en si mismo es $r$-conector. Defina
$$Q=\{x\in \RR^E\colon x(\delta^-(S))\geq 1, \forall\, \emptyset \neq S\subseteq V\setminus \{r\}, x\geq 0\}$$
El objetivo de este problema es demostrar que $Q=\text{conv}\{\chi^F\colon F\text{ es $r$-conector}\}+\RR_+^E$. Es fácil ver (no lo demuestre) que demostrar la igualdad anterior equivale a demostrar que $Q$ es un poliedro entero. Para hacerlo, demostraremos que el sistema que define a $Q$ es TDI (eso es suficiente pues el vector lado derecho es entero).

Sea $c\in \ZZ^E$ tal que el problema $(P)=\min \{c^Tx\colon x\in Q\}$ tenga solución finita. 
\begin{enumerate}[(a)]
\item (5 puntos) Escriba el dual $(D)$ del problema $(P)$, en variables $(y_S\colon \emptyset \neq S \subseteq V\setminus \{r\}$).
\item (15 puntos) Llame $y^*$ a una solución óptima de $(D)$ que minimice el potencial $\Psi(y)=\sum_{S} y_S |S||V\setminus S|$. (Existe pues $(P)$ tiene solución finita, y hay un número finito de vértices en el poliedro dual). Demuestre que $y^*$ tiene soporte laminar (es decir, que los conjuntos asociados a coordenadas no nulas de $y^*$ son una familia laminar).\\
\textbf{Indicación:} Al hacer el descruce, considere un par de conjuntos $A$ y $B$ intersectantes y asegúrese que $A\cap B$ y $A\cup B$ indiquen coordenadas del vector $y$.
Use (aunque no la haya hecho) el ejercicio ($f$) de la parte I. 
\item (10 puntos) Concluya que el sistema que define a $Q$ es totalmente dual integral.\\
\textbf{Indicación:} Use el resultado del ejercicio ($e$) de la parte I, aunque no lo haya hecho en su tarea.
\end{enumerate}



	\end{document}
	


