% BEGIN_FOLD preamble

\documentclass{article}
\usepackage[activeacute,spanish]{babel}
\usepackage{lmodern}
\usepackage[T1]{fontenc}
\usepackage{url,multirow}
\usepackage[centertableaux,smalltableaux]{ytableau}
\usepackage[utf8]{inputenc}
\usepackage{cancel, comment}
\usepackage[left=2cm,top=1.5cm,right=2cm, bottom=1.5cm,letterpaper, includeheadfoot]{geometry}
\usepackage{amssymb, amsmath, amsthm, mathtools}
\usepackage{graphicx}
\usepackage{hyperref}
\hypersetup{
	colorlinks,
	linkcolor={red!50!black},
	citecolor={blue!50!black},
	urlcolor={blue!80!black}
}

\usepackage[prependcaption,textsize=tiny,,textwidth=5cm]{todonotes}
\newcommand{\js}[1]{\todo[inline,linecolor=red,backgroundcolor=red!25,bordercolor=red]{jsoto. #1}}

\usepackage{paralist}
\usepackage{contour}
\usepackage{algorithm, algorithmic}

%%%Definiciones
\newcommand{\dis}{\displaystyle}
\newcommand{\IV}[1]{[\![#1]\!]} %Iverson

\newcommand{\E}{\mathcal{E}}
\newcommand{\V}{\mathcal{V}}

\def\multiset#1#2{\ensuremath{
		\mathchoice{\left(\kern-.3em\left(\genfrac{}{}{0pt}{}{#1}{#2}\right)\kern-.3em\right)}}
	{\big(\!\binom{#1}{#2}\!\big)}{\big(\!\binom{#1}{#2}\!\big)}{\big(\!\binom{#1}{#2}\!\big)}}
\DeclareRobustCommand{\sbinom}{\genfrac{[}{]}{0pt}{}}
\DeclareRobustCommand{\lbinom}{\genfrac{\{}{\}}{0pt}{}}
\newcommand{\fallfac}[2]{{#1}^{\underline{#2}}}
\newcommand{\risefac}[2]{{#1}^{\overline{#2}}}

\newcommand{\defn}[1]{\textit{\textsc{[Def]\,}}\textbf{#1}\\[5pt]\indent}
\newcommand{\nin}{\noindent}
% macros
\newcommand{\QQ}{\mathbb Q}
\newcommand{\RR}{\mathbb R}
\newcommand{\NN}{\mathbb N}
\newcommand{\ZZ}{\mathbb Z}
\newcommand{\FF}{\mathbb F}
\newcommand{\CC}{\mathbb C}
\newcommand{\EE}{\mathbb E}
\DeclareMathOperator{\COM}{COM}
\DeclareMathOperator{\com}{com}
\DeclareMathOperator{\ord}{ord}
\DeclareMathOperator{\DFJ}{DFJ}
\DeclareMathOperator{\EUL}{EUL}
\DeclareMathOperator{\WCOM}{WCOM}
\DeclareMathOperator{\wcom}{wcom}
\newcommand{\sop}{\operatorname{sop}}
\newcommand{\conv}{\operatorname{conv}}
\newcommand{\Ch}{\operatorname{Ch}}

%Teoremas, Lemas, etc.

\theoremstyle{plain}
\newtheorem{teo}{Teorema}
\newtheorem{lem}[teo]{Lema}
\newtheorem{prop}{Proposici\'on}
\newtheorem{cor}[teo]{Corolario}
\newtheorem{cor*}{Corolario}
\theoremstyle{definition}
\newtheorem{defi}[teo]{Definici\'on}
\newtheorem{eje}[teo]{Ejemplo}
\newtheorem{ejeres}[teo]{Ejercicios resueltos}
\newtheorem{ejere}[teo]{Ejercicio resuelto}
\newtheorem{ejes}[teo]{Ejemplos}
\newtheorem{ejer}[teo]{Ejercicio}
\newtheorem{prob}[teo]{Problema}
\newtheorem{obs}[teo]{Observaci\'on}
\newtheoremstyle{Azul}
{\topsep}   % ABOVESPACE
{\topsep}   % BELOWSPACE
{\color{blue}}  % BODYFONT
{0pt}       % INDENT (empty value is the same as 0pt)
{\color{blue}\bfseries} % HEADFONT
{.}         % HEADPUNCT
{5pt plus 1pt minus 1pt}  % HEADSPACE. `plain` default: {5pt plus 1pt minus 1pt}
{}          % CUSTOM-HEAD-SPEC
\theoremstyle{Azul}
\newtheorem*{comm}{Comentario}
\newcommand{\commento}[1]{\noindent{\color{blue}#1}\vspace*{-3pt}}

% fin macros

\usepackage{fancyhdr}
\pagestyle{fancy}
\fancypagestyle{plain}{%
\fancyhf{}
\lhead{\footnotesize\itshape\bfseries\rightmark}
\rhead{\footnotesize\itshape\bfseries\leftmark}
}

% END_FOLD
\begin{document}
% BEGIN_FOLD encabezado
\setlength{\headheight}{14pt}
\fancyhead[L]{Facultad de Ciencias F\'isicas y Matem\'aticas}
\fancyhead[R]{Universidad de Chile}
\vspace*{-1.2 cm}
\begin{minipage}{0.6\textwidth}
	\begin{flushleft}
		\hspace*{-0.5cm}\textbf{MA4702. Programación Lineal Mixta 2020.}\\
		\hspace*{-0.5cm}\textbf{Profesor: José Soto.}\\
	\end{flushleft}
\end{minipage}
\begin{minipage}{0.36\textwidth}
	\begin{flushright}
		\includegraphics[scale=0.15]{fcfm.pdf}
	\end{flushright}
\end{minipage}
\bigskip
%Fin encabezado

% END_FOLD
\newif\ifsol
\soltrue
\solfalse

\begin{center}
  \LARGE \textbf{Tarea 5}.\\
\end{center}
\bigskip

\noindent\textbf{Fecha entrega}: Viernes 24 de Julio, 23:59. Por u-cursos. (Se recomienda entregar el Miércoles para acelerar corrección)
\subsection*{Instrucciones:} 
\begin{enumerate}
\item \textbf{Extensión máxima}: Entregue su tarea en a lo más \textbf{6 planas}.
    \item \textbf{Formato:} La tarea debe entregarse en formato pdf, con fondo de un solo color (blanco de preferencia), letra legible en manuscrito y clara. (No se aceptarán documentos tipeados o generados por computador, pero si tiene alguna manera de escribir en manuscrito directamente de manera digital lo puede hacer).
    Si desarrolla su tarea en papel, entréguelo escaneados o en fotos de alta calidad, via ucursos.    \item \textbf{Tiempo de dedicación:} El tiempo estimado de \emph{desarrollo} de la tarea es de 2.5 horas de dedicación. Esto no considera el tiempo de estudio previo, el tiempo dedicado en asistir a cátedras y auxiliares, ni el tiempo para \emph{ponerse al día}. 
     Tendrá un plazo de 7 días para entregarlo. No espere hasta el último momento para escanear o fotografiar adecuadamente su tarea y cambiarla al formato solicitado (pdf). Entregue con suficiente anticipación a la hora límite.
    \item \textbf{Revisión:} Se podrá descontar hasta 1 punto en la nota final por falta de formato o extensión.
\item Esta tarea está pensada para ser hecha en forma individual.
    \end{enumerate}
    
\subsection*{Definiciones:}
Llamamos rango de Chvátal de un poliedro $P$, y lo denotamos $\Ch(P)$, al valor mínimo $k$ tal que la $k$-ésima cerradura de Chvátal,  $P^{(k)}$ es igual a $\conv(P\cap \ZZ^n)$. Por ejemplo, si $P$ es integral, $\Ch(P)=0$.
 
Si $\alpha^Tx\leq \beta$ es una desigualdad válida para $\conv(P\cap \ZZ^n)$, llamamos rango de Chvátal de la desigualdad en $P$ al menor valor $k$ tal que la desigualdad es válida para $P^{(k)}$

\subsection*{Ejercicios:} Cada uno vale 1.5 puntos.

\begin{enumerate}[(a)]
\item Edmonds demostró que para todo grafo $G=(V,E)$, 
$\conv(\chi^F\colon F \text{ matching perfecto de $G$})$ es igual al poliedro $$P =\{x\in\mathbb{R}^E_+\colon x(\delta(v))=1, \forall v\in V; x(\delta(U))\geq 1, \forall U\subseteq V, |U|\text{ impar}\}$$
Decimos $G$ es cúbico si el grado de cada vértice es 3. Es fácil ver (no lo demuestre) que en este caso $G$ tiene un número par de vértices.

Decimos que $G$ es 2-arista-conexo si para todo conjunto de vértices $U$, $\emptyset\neq U\neq V$, $|\delta(U)|\geq 2$. Use todo lo anterior para demostrar el siguiente teorema de Petersen.

Si $G$ tiene una cantidad par de vértices, y es 2-arista-conexo y cúbico, entonces $G$ tiene un matching perfecto.

\textbf{Solución}\\

Como el politopo $P$ es igual a la envoltura convexa de las indicatricez de matchings perfectos de $G$ nos queda demostrar que $P$ es no vacío. Para probar lo anterior utilizaremos el hecho de que el grafo es cúbico (de cada vertice salen exactamente tres arcos) y propondremos una solución simétrica con grado 1 en cada vértice, para esto usaremos $\hat{x}=\frac{1}{3}$, como el grafo es cúbico se tiene que $\hat{x}(\delta(v))=1$. Nos queda probar que $\hat{x}(\delta(U))\geq 1, \; \forall U \subseteq V, |U| \;\text{impar}$, para esto utilizaremos el hecho de que $\sum_{v\in U}|\delta(v)| = 2|E(U)|+|\delta(U)| = 3|U|$, esto implica que $|E(U)|=\frac{3|U|-|\delta(U)|}{2}$, por lo que si $|U|$ es impar $|\delta(U)|$ debe ser impar y como $G$ es 2-arista-conexo se tiene que $|\delta(U)|\geq 2$, por ende $|\delta(U)|\geq 3$ si $U$ es impar, con esto se tiene que $x(\delta(U))=\frac{1}{3}|\delta(U)|\geq 1$, luego $P$ es no vacío y por tanto tiene un matching perfecto.

\item 
Sea $k\geq 0$ entero y sea $Q$ un poliedro en $\RR^2$ tal que $Q\supseteq \{(0,0),(0,1),(1/2,k)\}$ y $Q\cap \ZZ^2=\{(0,0),(0,1)\}$. Pruebe que $\Ch(Q)\geq k$.

\textbf{Indicación:} Use inducción.

\textbf{Solución}\\

Probaremos esto por inducción. Si $k=1$ se tiene que $(1, 1/2)$ no es punto entero y no pertenece a la combinación convexa de $\{(0,0), (0,1)\}$, por lo que se requiere de al menos una cerradura de Chvátal para eliminarlo, por lo que $Ch(Q)\geq 1$. Para $k>1$, probaremos que si $\conv(\{(0,0), (0,1), (k, 1/2)\})\subseteq Q$, entonces, $\conv(\{(0,0), (0,1), (k-1, 1/2)\})\subseteq Q^{(1)}$, de hecho basta probar que $(k-1, 1/2)\in Q^{(1)}$ ya que $\{(0,0), (0,1)\}$ pertenence a todas las cerraduras de Chvátal ya que está en $Q\cap \mathbb{Z}^{2}$. \\
Consideremos el siguiente corte de Chvátal $a^{T}x\leq \lfloor b \rfloor$, obtenido de redondear el lado derecho de una desigualdad valida $a^{T}x\leq b$ de $Q$, con $a\in \mathbb{Z}^{2}$. Como $(0,0)$ satisface $a^{T}x\leq b$ se tiene que $b\geq 0$. Ahora consideraremos los siguientes dos casos:

\textbf{Caso 1 ($a_{1}>0$):} Como $a_{1}$ es entero positivo,  $a_{1}\geq b-\lfloor b \rfloor$, por tanto,  $\lfloor b \rfloor \geq b-a_{1} \geq ka_{1}+\frac{a_{2}}{2}-a_{1}=(k-1)a_{1}+\frac{a_{2}}{2}$, luego $a_{1}x_{1}+a_{2}x_{2}\leq \lfloor b \rfloor$ es valida para $(k-1, 1/2)$. \\
\textbf{Caso 2 ($a_{1}\leq 0$):} Como $(0,1)$ satisface $a_{1}x_{1}+a_{2}x_{2}\leq b$ se tiene que $a_{2}\leq b$, además, como $b\geq 0$ se cumple $a_{2}/2\leq\lfloor b \rfloor$. Debido a que $a_{1}\leq 0$ se tiene que $(k-1)a_{1}\leq 0$ puesto que $k>1$, por tanto, se cumple que $(k-1)a_{1}+\frac{a_{2}}{2}\leq \frac{a_{2}}{2}\leq \lfloor b \rfloor$.\\
En consecuencia, el punto $(k-1, 1/2)$ satisface todos los cortes de Chvátal por lo que $\conv(\{(0,0), (0,1), (k-1, 1/2)\})\subseteq Q^{(1)}$. Finalmente, por la hipótesis de inducción $\conv(\{(0,0), (0,1), (k, 1/2)\}$ tiene al menos rango $k$ y por ende $Q$ también.

\item Considere una mochila que soporta un peso máximo de $W$ y $n$ objetos, donde el $i$-ésimo objeto tiene peso $0< a_i\leq W$ (no necesariamente enteros).

Decimos que un conjunto $S\subseteq [n]$ es un packing, si los objetos indexados por $S$ caben en la mochila, es decir $\sum_{i\in S}a_i \leq W$. Decimos que $C\subseteq [n]$ es un cover, si $C$ no es un packing. Decimos que $C$ es un cover minimal si $C$ es un cover pero $C\setminus\{j\}$ no es cover para todo $j\in C$.

La envoltura convexa de las indicatrices de todos los packings es $P\cap \ZZ^n$ donde $P$ es el polítopo de mochila siguiente:
$$P=\{x\in \RR^n\colon \sum_{i=1}^na_ix_i \leq W; x_i \geq 0, \forall i\in [n], x_i\leq 1, \forall i\in [n]\}$$

Notamos que si $C$ es un cover minimal entonces todo packing puede contener a lo más $|C|-1$ elementos de $C$. En particular, la desigualdad de cover 
$$\sum_{i\in C}x_i \leq |C|-1$$ 
es válida para $P\cap \ZZ^n$.

Demuestre que para todo $C$ cover, la desigualdad de cover asociada tiene rango de Chvátal 1 en $P$ (es decir, es un corte de Chvátal de $P$).

\textbf{Indicación:} Encuentre una combinación cónica adecuada de las desigualdades de $P$.

\textbf{Solución}\\

 Para todo $i\in C$ multiplicamos por $\frac{W+1-a_{i}}{W+1}$ la desigualdad $x_{i}\leq 1$, para todo $i\notin C$ multiplicamos por $\frac{a_{i}}{W+1}$ la desigualdad $-x_{i}\leq 0$  y multiplicamos por $\frac{1}{W+1}$ la desigualdad $\sum_{i\in[n]}a_{i}x_{i}\leq W$, luego sumamos obteniéndose la siguiente desigualdad:

\begin{align*}
	\sum_{i\in C}\frac{W+1-a_{i}}{W+1}x_{i}+\sum_{i\notin C}\frac{-a_{i}}{W+1}x_{i}+\sum_{i\in[n]}\frac{a_{i}}{W+1}x_{i}&\leq |C|-\frac{1}{W+1}\sum_{i\in C}a_{i}+\frac{W}{W+1}\\
	\sum_{i\in C}x_{i}&\leq |C|+\frac{W-\sum_{i\in C}a_{i}}{W+1}
\end{align*}

Como $C$ es un cover se tiene que $W-\sum_{i\in C}a_{i}<0$ y como es minimal $\sum_{i\in C}a_{i}-a_{j}\leq W$ para todo $j\in C$, así $\sum_{i\in C}a_{i}-W\leq a_{j}\leq W$, por lo que $\frac{\sum_{i\in C}a_{i}-W}{W+1} \in (0,1)$, finalmente si tomamos cajon inferior se tiene $\sum_{i\in C}x_{i}\leq |C|-1$. 

\item Sea $G=(V,E)$ un grafo. Un conjunto estable de $G$ es un conjunto de vértices $S\subseteq V$ tal que $E(S)=\emptyset$. Definamos 
$$P_G=\{x\in \RR^E\colon x_u+x_v\leq 1, \forall uv\in E; x_v\geq 0, \forall v\in V\}$$
No es difícil probar que $P_G$ es una formulación para el conjunto de las indicatrices de conjuntos estables, es decir
$$\conv(\chi^S\colon S\text{ conjunto estable de $G$})=P_G\cap \ZZ^V.$$

Un conjunto $K$ de vértices es un clique de $G$ si $E$ contiene todas las aristas entre pares de vértices de $K$. Claramente, si $S$ es un conjunto estable entonces para todo clique $K$, $S$ contiene a lo más un vértice de $K$. Esto prueba que para todo $K$ clique, la desigualdad de clique:
$$\sum_{v\in K}x_v \leq 1$$
es válida para $P_G\cap \ZZ^V$.

Demuestre por inducción que si $K$ es un clique con $|K|\leq k$ entonces la desigualdad de clique de $K$ tiene rango de Chvátal a lo más $k-2$ en $P_G$.

Bonus: Tendrá 0.5 puntos adicionales que puede agregar a cualquier tarea si demuestra que el rango de Chvátal es en realidad $O(\log k)$. (Esta pregunta es binaria, no puede estar explicada a medias)

\textbf{Solución}\\
Probaremos que el rango de Chvátal de la desigualdad de clique sobre un clique con $k$ vértices es a lo más $k-2$ por inducción en $k$. Cuando $k=2$, tenemos que $x_{u}+x_{v}\leq 1$ con $(u,v)=K \in E$, desigualdad que ya está incluida en $P_{G}$, por lo que su rango es 0, satisfaciéndose el caso base. Consideremos un clique $K$ de tamaño $k$ para algún $k\geq 3$, luego por hipótesis de inducción para todo subclique $K'\subseteq K$ de tamaño $k-1$ se cumple $\sum_{v\in K'}x_{v}\leq 1$. Luego, si sumamos todas esas $k$ desigualdades (ya que existen $k$ subcliques de tamaño $k-1$) y dividimos por $k-1$ se obtiene la desigualdad $\sum_{v\in K}x_{v}\leq \frac{k}{k-1}$. Finalmente, como $k\geq 3, \lfloor\frac{k}{k-1}\rfloor=1$, obteniéndose la desigualdad clique que buscamos.

\end{enumerate}

\end{document}
	


